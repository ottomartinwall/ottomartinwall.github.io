\subsection{Instuderingsfrågor}

\paragraph{1. Vad är routing?/Hur går det till?}
Routing är vad datorer använder för att skicka och ta emot information av varann. \textit{Routinglistor/tabeller} och \textit{Default Gateway} används inom denna process. Processen ser ungefär ut såhär när en dator försöker kommunicera med en annan:
\begin{enumerate}
	\item Hosten kontrollerar först om sitt paket ska till en adress på det lokala nätverket eller inte, genom att kolla igenom sin routingtabell.
	\item Om paketet ska till en lokal adress så skickas det dit, men annars skickas det till den angivna \textit{Default Gateway} (routern för det lokala nätverket)
	\item Routern tar emot paketet.
	\item Routern läser av om paketet ska skickas vidare.
	\item Routern kollar igenom sin \textit{routingtabell} och väljer vilken väg som är bäst.
	\item Finns det en route så skickas paketet.
	\item Finns det ingen route så skickas paketet till routerns \textit{Default Gateway}, vilket är en annan router.
	\item Finns det ingen default gateway så slängs paketet.
	\item steg 3, 4, 5, 6 och 7 repeteras tills paketet är hos mottagaren eller har slängts.
\end{enumerate}

\paragraph{2. Vad är en nätmask?/Förklara varför det används}
En nätmask används för att ange vilka bytes i en \textit{IP adress} som ägnas åt \textit{Nät-ID} respektive \textit{Dator-ID}, så att datorn kan läsa av adressen rätt.

\paragraph{3. Vad är subnetting?/Förklara varför det används}
Tanken var ursprungligen att varje enhet skulle ha en egen fast \textit{IP-adress} i standarden \textit{IPv4}. Detta blev snabbt orimligt i och med internets expansion. Klass A var till för företag/organisationer och kunde ha 16 miljoner enheter under samma subnät men det kunde endast finnas 126 sådana i världen. Idag så används istället \textit{lokala IP-adresser} i ett nätverk, som sedan går mot en router som översätter de lokala adresserna till en gemensam adress som används för att kommunicera med resten av internet. Detta kommer tillslut inte heller att räcka, men det finns en standard vid namn \textit{IPv6} som vi sakta byter till som har \textbf{\textit{många}} fler möjliga adresser. Subnät används idag inom ett lokalt nätverk för att t.ex kategorisera in olika divisioner inom ett företag.

\paragraph{4. Vilka tre typer av routeanslutingar finns det?}
 

\paragraph{5. Nämn vilka sex routetyper det finns (6 st)}
\paragraph{6. Joakim ska skicka en fil till en kollega.(Samma nätverk) Hur går processen till? [E-A]}

Enligt sättet som beskrivs i fråga 1, så kan vi gå igenom stegen.

\begin{enumerate}
	\item Hosten kontrollerar om paketet ska skickas till en ip-adress på det lokala nätverket.
	\item Om det ska det, så skickas den dit.
\end{enumerate}

Filen kommer att skickas med hjälp av FTP eller SFTP. Filströmmen går igenom en switch som är kopplad mot båda enheterna.

\paragraph{7. Företaget du arbetar på använder mer enheter än vad deras klass C nätverk kan tilldela. Varför kan inte nätverket tilldela fler enheter och hur gör du för att lösa problemet?}

De kan byta till klass A. Då vi idag ändå använder lokala nätverk med routers ut mot resten av internet, så kan vi använda Klass A utan problem. Troligtvis har inte företaget mer än 126 lokala nätverk som de vill ha innanför samma yttre router, men om de har det så kan de även ta Klass B.

En alternativ lösning är att skapa fler subnät.

Klass C ägnar bara en byte åt dator-IDn, vilket resulterar att 128 finns, och 2 går åt annat så 126 är tillgängliga.

\paragraph{8. Förklara vad VPN är, hur det fungerar och varför det är bra att använda VPN både privat och företagsrelaterat.}

\textit{VPN (Virtual Private Network)} förlänger ett privat nätverk, genom att användaren kan koppla mot ett privat nätverk över internet och sedan kommunicera med resten av internet som att användaren befann sig inom det privata nätverket. Kopplingen brukar vara krypterad så att datan inte kan läsas av.

VPNer brukar användas till att t.ex nå nätverkskopplade resurser så som en NAS i hemmet eller kontoret. Det finns även VPN-tjänster så som ExpressVPN, som tillåter dig att koppla mot privata nätverk i olika länder för att ta sig runt \textit{geo-blocking}, vilket i praktik kan tillåta en att se andra serier på Netflix som inte finns i sitt hemland.

Två protokoll som används mycket idag är OpenVPN och Wireguard, varav det andra är nyare, säkrare och snabbare (generellt sett).

Fungerar DETTA?!?!?! HOPPAS DET IALLAFALL!!!!!!!

TEST 2!!!!!!!!!!
















