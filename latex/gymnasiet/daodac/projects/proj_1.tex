\subsection{Checkpoint}

\paragraph{1. Förklara}
\begin{itemize}
	\item \textbf{Genväg (Shortcut):} En genväg är en länk till den riktiga filen. Den kan användas på t.ex skrivbordet för att kunna öppna ett program som egentligen finns i en annan map.
	\item \textbf{Ljudvolym:} Ändring av ljudvolymen på datorn styr ljudeffekten hos kopplade högtalare.
	\item \textbf{Åtgärdscentret:} Detta kan öppnas genom en meddelande-ikon längst ned till höger på taskbar. Där finns snabba inställningar om t.ex bluetooth, wifi, ljusstyrka, stör-ej-funktion, mm. Meddelanden kan också komma hit.
\end{itemize} 

\paragraph{2. Vart finns de flesta inställningarna?}
I kontrollcentret (Control Center).

\paragraph{3. Hur gör man för att starta ett program som administratör?}
Högerklicka, och sedan klicka 'open as administrator'.

\paragraph{4. Vad gör windows update?}
Windows Update skannar efter, laddar ner och installerar nya uppdateringar och drivrutiner som behövs för datorn.

\paragraph{5. Vad skyddar datorn mot virus?}
Om inget tredjepartsprogram används så står Windows Defender för uppgiften. Det söker igenom filer som laddas ner och det kan t.ex skanna datorn utefter existerande virus.

\paragraph{6. Var i windows 10 kan man ändra vilka WIFI-nätverk som ska anslutas till automatiskt?}
I vanliga \textit{Settings} $\rightarrow$ \textit{Network and Internet} \textit{(I Windows 11 åtminstone)}.

\paragraph{7. Vilka tre typer av nätverksplatser finns det?}
\textit{Private Network}, \textit{Work Network}, \textit{Public Network}.

\paragraph{8. Nämn minst tre olika exempel på molntjänster}
\textit{Google Drive}, \textit{iCloud}, \textit{Dropbox}.

\paragraph{9. Vad kan man göra om man har ett program som har fryst sig?}
Tvångsavsluta det i \textit{Task Manager}.

\paragraph{10. Inom vilken tid måste man aktivera windows 10?}
Det måste aldrig göras. Konsekvensen av att inte aktivera är att några funktioner inte är tillgängliga (som inte egentligen är av så stor vikt) och att det står längst nere i högra hörnet att windows inte är aktiverat (men det går att ta bort).

\dotfill

\paragraph{1. Var i windows anger man vilka datakällor (SQL-server) som skall länkas till datorn?}
\textit{ODBC}, vilket finns i \textit{Control Panel}.

\paragraph{2. Hur får du fram dolda filer?}
\textit{File Manager} $\rightarrow$ \textit{Show} $\rightarrow$ \textit{Hidden Items}.

\paragraph{3. Vilka tre typer av datavirus finns det? Förklara 1 typ mer utförligt}

\begin{itemize}
	\item \textbf{\textit{Computer Virus}:} Detta är ett program som råkar köras av användaren, som sedan kan sprida sig och köra processer utan att upptäckas eller skapa någon märkvärd prestandaskillnad.
	\item \textit{\textbf{Trojan Virus}}
	\item \textit{\textbf{Worm}}
\end{itemize}

\paragraph{4. Vilken åtgärd kan vara lämplig om man misstänker fel på internminnet?}
Kör \textit{memtest86} och se om några error uppkommer, och om det gör det så testa alla ramstickor var för sig och se vilken eller vilka ramstickor som det är fel på.

\paragraph{5. Vilka tre filsystem för hårddiskar kan användas i windows 10?}
\begin{itemize}
	\item NTFS
	\item FAT32
	\item FAT16
\end{itemize}

\paragraph{6. Vilken mapp i profilen innehåller sparade inställningar för olika program?}
\textit{Appdata}

\paragraph{7. Vilka är de fem behörighetsnivåerna man som standard kan ange för filer?}
\textit{read}, \textit{write}, \textit{write and execute}, \textit{modify}, \textit{full control}

\paragraph{8. Förklara skillnaden mellan Partition och Volym. Ange exempel på situationer där du använder dem.}

\paragraph{9. Du laddar ner ett program som kräver en minimum av prestanda och en rekommenderad prestanda på din dator. Hur kan du veta om din dator uppfyller kraven och kan du checka i windows att din dator hanterar programmet och inte blir överbelastat?}

\paragraph{10. En person kommer till dig och vill ha hjälp att fixa sin dator. Personen berättar att datorn fungerade alldeles utmärkt tills igår. Personen stängde ner datorn men den behövde uppdateras innan den stängdes av. Vad är problemet och hur kan du fixa datorn?}

\newpage
\subsection{Instuderingsfrågor}

\paragraph{1. Förklara}
\begin{itemize}
	\item \textbf{Genväg (Shortcut):} En genväg är en länk till den riktiga filen. Den kan användas på t.ex skrivbordet för att kunna öppna ett program som egentligen finns i en annan map.
	\item \textbf{Ljudvolym (Volume):} Ändring av ljudvolymen på datorn styr ljudeffekten hos kopplade högtalare.
	\item \textbf{Åtgärdscentret:} Detta kan öppnas genom en meddelande-ikon längst ned till höger på taskbar. Där finns snabba inställningar om t.ex bluetooth, wifi, ljusstyrka, stör-ej-funktion, mm. Meddelanden kan också komma hit.
\end{itemize}

\paragraph{2. Vad för typ av fil?}
\begin{enumerate}
	\item \textbf{\textit{.exe}:} EXE eller .exe (förkortning för engelskans executable) är en filändelse i bland annat operativsystemen DOS, OS/2 och Microsoft Windows. Filer med denna ändelse är exekverbara, det vill säga de är program som går att köra.
	\item \textbf{\textit{.txt}:} txt är en filändelse på textfiler med enbart enkel text utan formatering, annat än sådan som är avsedd att direkt tolkas av läsaren. En. txt-fil är också uttryckligen avsedd att läsas, inte att tolkas av något datorprogram.
	\item \textbf{\textit{.docx}:} docx ett Open XML-formaterat Microsoft Word dokument. Alla program kan inte läsa hela filformatet. och i vissa fall kanske ett program bara kan läsa delar av filen. Till exempel kan ett program läsa texten, men inte formateringen, av en fil som använder ett annat format än sitt eget.
	\item \textbf{\textit{.pdf}:} Det är ett format du ständigt använder i arbetet. Adobe Portable Document Format (mer känt som pdf) för att dokument skulle kunna visas och skickas på ett tillförlitligt sätt, oavsett program, maskinvara eller operativsystem.
	\item \textbf{\textit{.jpg}:} JPG, eller JPEG, är ett bildformat som främst används när det kommer till digitala foton. Bilder som sparas i JPG-format komprimeras destruktivt. Det innebär att bildens filstorlek minskas genom att datorn tar bort information från bilden.
\end{enumerate}

\paragraph{3. Förklara mapp och hur du skapar en.} En mapp är en samling/gruppering av filer som fungerar som en genväg. Mappen kan innehålla alla möjliga olika filtyper. Man skapar en mapp enkelt genom att högerklicka på en tom yta och sedan trycka på \textit{New Folder}.

\paragraph{4. Vart finns de flesta inställningarna?}
I \textit{Control Center}.

\paragraph{5. Var avinstallerar du ett program?}
I \textit{Control Panel} $\rightarrow$ Programs $\rightarrow$ Programs and Features $\rightarrow$ Uninstall or change a program. Det går också att avinstallera program i vanliga \textit{Settings}.

\paragraph{6. Vad gör windows update?}
Det är ett verktyg i \textit{Windows 10} som hanterar uppdateringar av operativsystemet, drivrutiner mm.

\paragraph{7. Vad skyddar datorn mot virus?}
\textit{Windows Defender} i \textit{Windows 10}.

\paragraph{8. Vilken säkerhetsåtgärd finns att använda mot förlust av data?}
Säkerhetskopieringar. Det kan vara automatiserade säkerhetskopieringar mot en molntjänst, eller till en NAS i hemnätverket, eller bara till en hårddisk i datorn.

\paragraph{9. Var i windows anger man vilka datakällor (SQL-server) som skall länkas till datorn?}
I \textit{ODBC Data Source Administrator}.

\paragraph{10. Kan man logga händelser i operativsystemet?}
Ja, i \textit{Event Viewer}.

\paragraph{11. Vad är en process?}
Det är ett aktivt program eller del av ett program, ett program som körs i bakgrunden eller en del av operativsystemet.

\paragraph{12. Hur gör man för att köra ett program som administratör?}
Högerklicka på och välj \textit{Run as administrator}.

\paragraph{13. Vart kollar du hur mycket av datorns resurser som används?}
Det kan ses i \textit{Task Manager} $\rightarrow$ \textit{Performance}.

\paragraph{14. Vilka tre typer av datavirus finns det?}
\begin{enumerate}
	\item \textbf{\textit{Computer Virus}:} Ofta ett litet program som kan infektera filer och sprida sig till andra datorer.
	\item \textbf{\textit{Trojan Virus}:} Dessa är program som användaren luras installera och starta, som sedan ger tillgång till datorn för skaparen av viruset.
	\item \textbf{\textit{Worms}:} Dessa sprider sig själva över nätverk och internet, och problemet med dem är ofta att de tar upp mycket bandbredd när de sprider sig fort, mer än att de egentligen sedan gör skada hos en dator.
\end{enumerate}

\paragraph{15. Vad är ett antivirusskydd?}
Ett antivirusskydd jämför koden hos program och filer mot existerande kända virus, trojaner och maskar. När och om den hittar dessa så kan det sätta dem i karantän eller radera dem.

\paragraph{16. I vilken mapp i windows lagras de flesta drivrutinerna?} De kan hittas i \textit{C:\textbackslash Windows\textbackslash System32}

\paragraph{17. I vilka 3 program hittar man de flesta funktionerna och program för att hantera maskinvara?}
I \textit{Control Panel}, vilket innehåller \textit{Device Manager} och \textit{Devices and Printers}.

\paragraph{18. Var kan man göra inställningar för de resurser som en maskinvara använder?} I \textit{Device Manager}.

\paragraph{19. Om man har installerat en ny drivrutin och datorn hänger sig och inte fungerar efter omstart, vad kan man göra då?} Klicka F8 vid omstart och sedan hitta och välj \textit{Last Known Good Configuration}. Detta startar datorn i just det.

\paragraph{20. Vilka tre filsystem för hårddiskar kan användas i windows 10?} \textit{FAT/FAT16} (kom från MS-DOS), \textit{FAT32}, \textit{NTFS} (mest avancerad och säker, används från Windows Vista och framåt).

\paragraph{21. Vilket av dessa är säkrast? (kopplat till fråga 20)} \textit{NTFS}.

\paragraph{22. Förklara skillnaden mellan Partition och Volym}
En Partition är en allokerad del av en disk. En volym kan bestå av flera partitioner (men det sker sällan), och det är vad som syns i Windows som C: t.ex.

\paragraph{23. Kan man krympa volymen med systemet i Windows 10?} Ja. 

\paragraph{24. Vart gör man inställningar för val startoperativsystem?}
\textit{System Properties} $\rightarrow$ \textit{Advanced} $\rightarrow$ \textit{Startup and Recovery} $\rightarrow$ \textit{Default Operating System}.

\paragraph{25. Hur kommer man åt den avancerade startmenyn?}
Genom att trycka \textit{F8} vid upstart/omstart av datorn, innan Windows startbild visas. I praktik så kan man spamma \textit{F8} så att man inte missar den rätta tidsperioden att trycka knappen.

\paragraph{26. Vad kan vara en lämplig åtgärd om datorn fungerat bra fram tills igår när ett flertal uppdateringar installerats?}
Sök efter \textit{Create a restore point} och öppna, sedan klicka \textit{System Restore}. Detta kräver att man innan har sparat en \textit{restore point}.

\paragraph{27. Nämn tre verktyg för att studera datorns resursanvändning och prestanda?}
\textit{Task Manager}, \textit{HWINFO64 (third party)}, \textit{HWMonitor (third party)}.

\paragraph{28. Vilken åtgärd kan vara lämplig om man misstänker fel på internminnet?}
Kör \textit{memtest86} i 12 timmar (eller kortare om ett error kommer fram). Om ett error kommer fram, testa samma sak med varje enskild ramsticka för sig självt för att ta reda på vilken eller vilka som skapar problemen. Byt ut de som skapar fel.

\paragraph{29. I vilket program i windows hanterar man alla nätverksanslutningar i datorn?}
I windows 11 åtminstone, så är det: \textit{settings} $\rightarrow$\textit{Network and internet}.

\paragraph{30. Vilka tre typer av nätverksplatser finns det?}
\textit{Home network}, \textit{Work network}, \textit{Public network}

\paragraph{31. Vilken åtgärd kan man göra i Windows 10 om man får problem med en nätverksanslutning?}
Checka att lösenordet är rätt, starta om nätverksanslutningen och testa igen och om det inte fungerar så starta om routern.

\paragraph{33. Vad menas med en molntjänst?}
En tjänst som körs på någon annans dator. Google Drive är ett exempel, som körs på googles servrar.

\paragraph{34. Nämn minst tre exempel på olika molntjänster}
Google Drive, iCloud, Dropbox.

\paragraph{35. Vad krävs för att kunna använda använda datorer i en hemgrupp?}
Nätverkstypen måste vara \textit{Home network}.

\paragraph{36. Inom vilken tid måste man aktivera Windows?}
Det behövs aldrig. Vissa funktioner är bara inte tillgängliga och ett \textit{Water mark} syns ofta i nedre högra hörnet (som går att få bort via lite trix)

\paragraph{37. Hur får du fram dolda filer?}
I \textit{File Explorer} $\rightarrow$ \textit{View} $\rightarrow$ \textit{Show} $\rightarrow$ \textit{Hidden items}.

\paragraph{38. Hur startar man aktivitetshanteraren? \textit{(Task Manager)}}
Högerklicka på windowsikonen nere i vänsta hörnet, och klicka \textit{Task manager}.

\paragraph{39. Vad menas med en process i Windows?}
Det är ett program eller en part av ett program som är aktivt och använder resurser.

\paragraph{40. Hur kan man få en uppfattning om hur väl datorns prestanda uppfyller kraven från de program och tjänster som körs i datorn?}
Checka de rekommenderade specifikationskraven för de program du undrar om. Det går även att testa dem och se hur det går.

\paragraph{41. Vad kan man göra om man har ett program som har "fryst" sig?}
Tvångsavsluta det i \textit{Task Manager}.

\paragraph{42. Hur gör för att köra ett program som administratör?}
Högerklicka och välj \textit{Open as administrator}.

\paragraph{43. Vilken mapp i profilen innehåller sparade inställningar för olika program?}
mappen \textit{Appdata}.

\paragraph{44. Var kan man ange Systemrättigheter (användarrättigheter)?}
Windows 10 Home har inte \textit{Local Group Policy Editor} där det görs. Pro eller Enterprise krävs. Men om man har det så är det där.

\paragraph{45. Vilket filsystem krävs för att man ska kunna ange behörigheter till filer och mappar?}
\textit{NTFS}.

\paragraph{46. Hur anger man en behörighet till en fil eller mapp?} Högerklicka på filen $\rightarrow$ \textit{Properties} $\rightarrow$ \textit{Security}. Där kan behörigheterna ändras.
 
\paragraph{47. Vilka är de sex behörighetsnivåerna man som standard kan ange för mappar?}
\textit{List Folders Contents}, \textit{Read}, \textit{Read and Execute}, \textit{Write}, \textit{Modify}, \textit{Full Control}. 

\paragraph{48. Vilka är de fem behörighetsnivåerna man som standard kan ange för filer?}
\textit{Read}, \textit{Read and Execute}, \textit{Write}, \textit{Modify}, \textit{Full Control}. 















