\subsection{Names and Nouns}

\subsubsection{Capital Letter}
\begin{important}[Capital Letter]
    Names of people, places, organizations, etc. always begin with a capital letter. In English, capital letters are also used for days, months and festive holidays.
\end{important}

Some examples:
\begin{itemize}
    \item I will watch that movie on Tuesday.
    \item What will you do on Christmas?
    \item I will soon move to Sweden.
    \item I'm looking forward to traveling to Spain in December.
\end{itemize}

\subsubsection{Titles}
\begin{important}[Titles]
    Personal titles begin with a capital letter.
    
    Titles of articles, books, movies etc. use a capital letters for each word except for articles\footnote{Articles are used in front of nouns to add meaning. They consist of the following words: \textit{a, an, the}.}, prepositions\footnote{Prepositions are words that show position or direction. Some examples are: \textit{up, down, around, among, with}.} and coordinating conjunctions\footnote{see \ref{Conjunctions}}. These three exceptions still have a capital letter if they are the first or the last word in a title.
\end{important}

Some examples of personal titles:
\begin{itemize}
    \item Professor
    \item Doctor
    \item Mr.
    \item Mrs.
    \item Ms.
\end{itemize}

Some examples of titles for articles, books, movies etc.
\begin{itemize}
    \item The Dog that Lived
    \item He Sailed on a Boat
    \item We Never Saw Him Again
    \item It's for the People
\end{itemize}

\subsubsection{Collective Nouns}
\begin{definition}[Collective Nouns]
    Collective nouns are nouns that describe a collection of things but are counted as a whole, like a \textit{team} or a \textit{group}. Such nouns can be used in the singular or in the plural, depending on whether the noun is seen as a whole or as a group of individual things (which is subjective).
\end{definition}
\label{Collective Nouns}

Some examples:
\begin{itemize}
    \item The crowd \textit{was} in a state of ecstasy. (\textit{singular})
    \item The crowd \textit{were} throwing stones. (\textit{plural})
    \item Group A \textit{is} a subset of group B. (\textit{singular})
\end{itemize}

\newpage
\subsection{Abstract Nouns}
\begin{definition}[Abstract Noun]
    An abstract noun is a word that refers to something that does not exist physically.
\end{definition}
Abstract nouns can often be recognized by their suffix, which usually is:
\begin{itemize}
    \item -tion: \textit{description, satisfaction}
    \item -ism: \textit{tourism, alcoholism}
    \item -ment: \textit{enjoyment, astonishment}
    \item -ness: \textit{happiness, rudeness}
    \item -ity: \textit{quality, reality}
\end{itemize}

\newpage
\subsection{The Passive Form (Not on the Test!)}
\begin{definition}[Passive Form]
    The passive form of a verb is created by using any form of be\footnote{this may be: \textit{be, am, is, are, was, were, being, been}} with a verb in past participle\footnote{see \ref{Regular and Irregular Verbs}}
\end{definition}
Passive form is often used when the object\footnote{see \ref{Subjects and Objects}} is unknown, or that the focus is on the action rather than the object. Here are an examples of a sentence with its verb in active form and passive form:

\begin{enumerate}
    \item Lucy \textit{writes} a speech. (\textit{active})
    \item A speech \textit{is written} by Lucy. (\textit{passive})
\end{enumerate}

\newpage
\subsection{Inversion}
\begin{definition}[Inversion]
    Inversion is when the order of the subject and verb is switched. It is usually S-V but then it becomes V-S.
\end{definition}
In english, sentences are usually written with with the subject before the verb, like \textit{She walked}. In some instances, this is reversed.

The most common scenarios where inversion is used is and an example with it is:
\begin{itemize}
    \item In questions: \textit{Did he really do that?}
    \item When a sentence starts with negative adverbs: \textit{Never had I seen such terrible behavior.}
    \item In expressions beginning with 'not': \textit{Not only does he drink but he also uses drugs.} 
\end{itemize}

\newpage
\subsection{Adjectives and Adverbs}
\begin{definition}[Adjectives and Adverbs]
    Adjectives describe and compare nouns and pronouns. Adverbs describe verbs.
\end{definition}

Some examples of adjectives are:

\begin{itemize}
    \item The cat is \textit{small}.
    \item The \textit{yellow} color fits this room.
\end{itemize}

Some examples of adverbs are:

\begin{itemize}
    \item He ran \textit{fast}.
    \item She pays \textit{well}.
\end{itemize}

\newpage
\subsection{Relative Clauses}
\begin{definition}[Clauses]
    A clause is a group of words that contains a subject and a predicate\footnote{a predicate says what the subject is or is doing.}
\end{definition}
An example of a clause is:

\begin{itemize}
    \item The cat is sleeping in the sun.
\end{itemize}

Clauses can be put together with the use of conjunctions (see \ref{Conjunctions}). An example of two main clauses being put together in a compound sentence is:

\begin{itemize}
    \item \textbf{She bought a new computer} \textit{and} \textbf{she purchased a used printer.}
\end{itemize}

\begin{definition}[Relative Clauses]
    A \textit{relative clause} is a \textit{subordinate clause}\footnote{a dependent clause that only has meaning in the context of a main clause to which it is connected.} that gives additional information about the contents of the main clause.
\end{definition}

Relative clauses are often introduced with a relative pronoun, which mainly are:

\begin{itemize}
    \item \textit{Who} 
    \item \textit{Whom} 
    \item \textit{Whose} 
    \item \textit{Which} 
    \item \textit{That} 
\end{itemize}

When defined this way, they are called a \textit{defining relative clause}. Some examples are:

\begin{itemize}
    \item The car \textit{which I drive} is very old. 
    \item Some children \textit{that were missing} have been found.
    \item My cousin \textit{who just called} had some really good news.
\end{itemize}

A relative clause can also be a \textit{non-defining relative clause}, which means that we already know who or what we are talking about when a relative pronoun is used to begin the relative clause. The non-defining relative clause is placed between commas. Some examples are:

\begin{itemize}
    \item The guy, \textit{who lives downstairs}, has got a kick-ass computer.
    \item That man, \textit{whose sister is quite fine}, is annoying.
\end{itemize}

The focus is therefore not on the non-defining relative clause, it is not necessary, but it gives additional information.


\newpage
\subsection{Verb tenses}
\subsubsection{The Present}
The present tenses are used on verbs that describe things going on right now, or are true now or always. They are also used for things that will happen in the future. These are the four present tenses in english:

\begin{enumerate}
    \item \textbf{Present simple} is the base form of the verb. It is used to express something that is true now or always.
    \item \textbf{Present continuous} is made up of \textit{any form of be\footnote{be, am, is, are, was, were, being, been} + ing-form} of the verb. It is used to express something that is happening right now or for a limited period of time.
    \item \textbf{Present perfect} is made up of \textit{have + past participle\footnote{see \ref{Regular and Irregular Verbs}}} of the verb. It is used for things that started in the past, but are still going ono now. It is also used for things that happened in the past but are still true now.
    \item \textbf{Present perfect continuous} is used in the same way as present perfect, but when we want to put more emphasis on the fact that something is still going on.
\end{enumerate}

Here's an example of four sentences, one for each present tense:

\begin{center}
    \resizebox{\columnwidth}{!}{
        \begin{tabular}{|c|c|c|c|}
            \hline
            \textbf{Present simple} & \textbf{Present continuous} & \textbf{Present perfect} & \textbf{Present perfect continuous} \\
            \hline
            \hline
            I play hockey. & I am playing hockey. & I have played hockey. & I have been playing hockey. \\
            \hline
        \end{tabular}
    }
\end{center}

\subsubsection{The Past}
The past tenses are used on verbs that describe things that have already happened, or to talk about things that could \textit{possibly} happen in the present or future. These are the four past tenses in english:

\begin{enumerate}
    \item \textbf{Past simple} is usually created by adding \textit{-ed} to the verb, but there are also many verbs with irregular forms. It is mainly used to express things that both started and ended in the past:
    \item \textbf{Past continuous} is made up of \textit{past tense of be\footnote{be, am, is, are, was, were, being, been} + ing-form} It is mainly used to describe what was going on when something else happened or around a certain time, for things that continued for some time.
    \item \textbf{Past perfect} is made up of \textit{had + past participle\footnote{see \ref{Regular and Irregular Verbs}}} of the verb. It is mainly used to describe things that started in the past and continued up to a certain point in the past. It is also used to show cause and effect.
    \item \textbf{Present perfect continuous} is made up of \textit{past perfect of any form of be + -ing form of the verb}. It is used in the same way as past perfect, but when we want to put more emphasis on the process.
\end{enumerate}

\begin{center}
    \resizebox{\columnwidth}{!}{
        \begin{tabular}{|c|c|c|c|}
            \hline
            \textbf{Past simple} & \textbf{Past continuous} & \textbf{Past perfect} & \textbf{Past perfect continuous} \\
            \hline
            \hline
            I played hockey. & I was playing hockey. & I had played hockey. & I had been playing hockey. \\
            \hline
        \end{tabular}
    }
\end{center}

\newpage
\subsection{Subject Verb Agreement}
\label{Subject Verb Agreement}
Subjects and verbs must agree in singular/plural in a sentence. Plural subjects must be connected with a plural verbs. If there are two or more singular subjects connected with \textit{and}, then they are treated as plural and will need a plural verb. If two or more singular objects are connected with \textit{or} then they are treated as singular and will need a singular verb. If singular and plural subjects are connected with \textit{or} then the subject closest to the verb dictate whether the verb is plural or singular. Collective nouns\footnote{see \ref{Collective Nouns}} can be treated as singular or plural, depending on what's the focus.

Some examples:
\begin{itemize}
    \item \textbf{She} \textit{walks} to school. (\textit{singular subject and verb})
    \item \textbf{They} \textit{walk} to school. (\textit{plural subject and verb})
    \item \textbf{Strength} and \textbf{fitness} \textit{are} important qualities in this competition. (\textit{two singular subjects connected with 'and' $ \Rightarrow $ plural verb})
    \item \textbf{The cat} \textit{or} \textbf{the dog} has taken the fish. (\textit{two singular objects connected with 'or' $ \Rightarrow $ singular verb})
    \item \textbf{The cat} \textit{or} \textbf{the dogs} have taken the fish. (\textit{subject closest to verb is plural and the subjects are connected with 'or' $ \Rightarrow $ plural verb})
\end{itemize}

\newpage
\subsection{Regular and Irregular Verbs}
\label{Regular and Irregular Verbs}
\subsubsection{Regular Verbs}
Regular verbs are conjugated to \textit{past simple} and \textit{past participle} by adding \textit{-ed} at the end of the word. Here are some examples:

\begin{center}
    \begin{tabular}{|c|c|c|}
        \hline
        Infinitive & Past Simple & Past Participle \\
        \hline
        \hline
        Pick & Picked & Picked \\
        \hline 
        Return & Returned & Returned \\
        \hline 
        Punch & Punched & Punched \\
        \hline 
        Question & Questioned & Questioned \\
        \hline 
    \end{tabular}
\end{center}

\subsubsection{Irregular Verbs}
Irregular verbs are verbs whose conjugation \textbf{do not} follow the typical pattern. There is no general rule for how to conjugate the irregular verbs unfortunately, but here are some examples of conjugated irregular verbs:

\begin{center}
    \begin{tabular}{|c|c|c|}
        \hline
        Infinitive & Past Simple & Past Participle \\
        \hline
        \hline
        Feel & Felt & Felt \\
        \hline 
        Fly & Flew & Flown \\
        \hline 
        Fit & Fit & Fit \\
        \hline 
        Have & Had & Had \\
        \hline 
    \end{tabular}
\end{center}

\newpage
\subsection{Apostrophe Rules}
We use apostrophes for contracting words and for possessive form.

Here are some examples for using it for contracting words:

\begin{itemize}
    \item It's quite easy. (\textit{It is})
    \item I must've forgotten it. (\textit{must have})
    \item We've been shopping all day. (\textit{We have})
\end{itemize}

Here are some examples for using it for possessive form:

\begin{itemize}
    \item The school's principal decided to close the school.
    \item The dog's bone is really large.
    \item Children's books.
    \item We need to retrieve the computers' internals.
\end{itemize}

Some other rules with apostrophes are that they are \textbf{not} used for decades or years:

\begin{itemize}
    \item The 1960s (\textit{Not 1960's})
    \item The 20s (\textit{Not 20's})
\end{itemize}

and they are also not used for acronyms:

\begin{itemize}
    \item UFOs (\textit{not UFO's})
\end{itemize}

\newpage
\subsection{Short Explanations of Prerequisites}
\subsubsection{Prepositions}
Prepositions are words that show position or direction. Some examples are: \textit{up, down, around, among, with}.

\subsubsection{Conjunctions}
\label{Conjunctions}
Conjunctions is a word that joins parts of a sentences, like clauses. Some examples are: \textit{for, since, yet, whom, so}.
\paragraph{Coordinating Conjunctions}
Coordinating conjunctions are conjunctions that join parts of equal importance, like: \textit{for, and, nor, but, or, so}. Some examples that do not join parts of equal importance are \textit{since, although after, because before, when while}.

\subsubsection{Subjects and Objects}
\label{Subjects and Objects}
A basic rule is that the subject is the person or thing doing something. The object is having something done to it. Both the subject and object can consist of multiple words. An object can be 'The man whose house my brother built' and a subject can be 'The cats'.

A \textbf{subject} is a noun, and there are three criteria for identifying subjects:
\begin{enumerate}
    \item Subject verb agreement: \textit{(see \ref{Subject Verb Agreement})}
    \item Position occupied: \textit{The subject typically immediately precedes the verb.}
    \item Semantic Role: \textit{It performs the action expressed by the verb.}
\end{enumerate}

Here are some examples of sentences where the subject is identified by being in bold and the connected verb is identified by being in italics.

\begin{itemize}
    \item \textbf{I} \textit{love} chocolate.
    \item \textbf{They} \textit{went} to the cinema.
    \item \textbf{The cats} \textit{like} her.
\end{itemize}

An \textbf{object} is a noun that comes after and is governed by a verb or a preposition. Some examples of objects in sentences are these:

\begin{itemize}
    \item I \textit{know} \textbf{him}. 
    \item \textit{Give} \textbf{her} the prize.
    \item Sit \textit{with} \textbf{them}.
\end{itemize}

where the italic words are the verbs or prepositions, and the bold words are the objects.

Here are some examples of sentences with both a subject and an object, where the subject is in bold and the object is in italics:

\begin{itemize}
    \item \textbf{He} paints \textit{a painting}.
\end{itemize}