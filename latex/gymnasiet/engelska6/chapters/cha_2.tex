\subsection{Old English Age}

\subsubsection{Name and Year}
Old english age, circa year 500 to 1100.

\subsubsection{The Situation in English Society}
England was invaded by Angle-Saxons.

\subsubsection{Influence in Litterature}
England was invaded by Angle-Saxons, who brought with them their language, culture and poetry. Not much was written down in text at this time, so most stories was learned and passed on verbally.

\subsubsection{Common Forms of Litterature}
Poetry and long stories about heroes and adventures.

\subsubsection{Famous Authors and Works}
Beowulf, which is about a scandinavian hero who fights monsters and dragons.

\subsubsection{\textit{One Famous Work Example}}
\textit{(same as above)} Beowulf, which is about a scandinavian hero who fights monsters and dragons.

\newpage
\subsection{Medieval Period}

\subsubsection{Name and Year}
Medieval Period, circa year 1100-1500.

\subsubsection{The Situation in English Society}
They were invaded by Normans (from France), so french began to be used in the upper class.

\subsubsection{Influence in Litterature}
Stories began touching more on religious themes than before, which may be the result of a shift from paganism (non-christian religions) to christianity during this time.

\subsubsection{Common Forms of Litterature}
Narrative style, heroic stories and stories with religious themes.

\subsubsection{Famous Authors and Works}
The Canterbury Tales. It is a collection of stories written by a group of pilgrims on their way to Canterbury Cathedral.

\subsubsection{\textit{One Famous Work Example}}
The Canterbury Tales. 24 stories (would've been 29). The story is incomplete. It was written by pilgrims traveling to Canterbury Cathedral.

\newpage
\subsection{Renaissance}

\subsubsection{Name and Year}
Renaissance, circa year 1500-1660.

\subsubsection{The Situation in English Society}
The Middle Ages was called the dark ages, so naturally what came after brought some technological leaps, trade and exploration. Stronger cultural links were made to other parts of the world. This period was also affected by the rule of Henry VIII, which was not a positive thing. He also split up with the Pope in Rome and therefore with Catholicism. 
Art and culture from Ancient Rome and Ancient Greece was brought back in the upper class.

\subsubsection{Influence in Litterature}
The printing press was invented during this time, so it became much easier to spread literature.

\subsubsection{Common Forms of Litterature}
Poetry, "metaphysical" poems (used very advanced, witty and far fetched literary deviceses), sonnet, plays (comedies and tragedies).

\subsubsection{Famous Authors and Works}
Romeo and Juliet, Hamlet \textit{(William Shakespeare)}

\subsubsection{\textit{One Famous Work Example}}
Romeo and Juliet, which is (shortly described) about a forbidden love.

\subsubsection{Comparison Between \textit{Julius Ceasar} and \textit{Romeo and Juliet}}
\textbf{1. Summary}
\begin{itemize}
	\item \textbf{Romeo and Juliet:} Romeo and Juliet is about a forbidden love between Romeo and Juliet, both from two different powerful families who's age old vendetta has erupted into bloodshed. Juliet fakes her death to be able to be with Romeo, but she can't reach him before he kills himself because he believes that she is dead. She then kills herself too.
	\item \textbf{Julius Ceasar:} Julius Ceasar is about the assassination of that said man. He gains too much power and to defend democracy, he has to be killed (which didn't result in anything better really).
\end{itemize}

\textbf{2. In what ways are the stories similiar?}
The stories are similiar in the way that both are very dramatic, both have people in great power, and both contain deaths.

\textbf{3. Can you detect a "typical" Shakespearean style in the stories, sort of like a similiarity in the themes? If so, how?}
I'd say that this question is quite similiar to question 2. I choose to respond with that same response.

\textbf{4. Which of the two stories would you like to read in its full length? Justify your answer.}
I don't want to read them. In my free time, I'd rather do other things.

\newpage
\subsection{Enlightenment}

\subsubsection{Name and Year}
Enlightenmnet, circa 1660-1789

\subsubsection{The Situation in English Society}
This was the \textit{Age of Reason}, where intelligence, knowledge and science was of importance. Religion had to take a step back.

\subsubsection{Influence in Litterature}
The litterature shifted to be about more real and more complex matter, compared to before. This was probably because intelligence and science was valued higher now than before.

\subsubsection{Common Forms of Litterature}
Satire, essays, letters, diaries. Political critism and social reality.

Novels also joined the game during this time, because writing and printing books became cheaper and therefore more widespread.

\subsubsection{Famous Authors and Works}
Gulliver's Travels \textit{(Jonathan Swift)}, Robinson Crusoe \textit{(Daniel Defoe)}.

\subsubsection{\textit{One Famous Work Example}}
Gulliver's Travels \textit{(Jonathan Swift)}

\newpage
\subsection{Romantic Age}

\subsubsection{Name and Year}
Romantic age, circa year 1789-1830.

\subsubsection{The Situation in English Society}
Before came common sense, reason and logic, but now society went back to dark times again. This period was all about feelings and non-existing creatures and monsters.

\subsubsection{Influence in Litterature}
Litterature was of course influenced by this shift in mentality, which resulted in that novels and poetry mostly had romantic, melancholic or supernatural themes.

\subsubsection{Common Forms of Litterature}
Novels, and now gothic novels (horror, mystery), became well established. Poetry was still common.

\subsubsection{Famous Authors and Works}
Frankenstein, or The Modern Prometheus \textit{Mary Shelley}, The Fall of the House of Usher \textit{(Edgar Allan Poe)}.


\subsubsection{\textit{One Famous Work Example}}
Frankenstein, or The Modern Prometheus \textit{Mary Shelley}

\newpage
\subsection{Victorian Age}

\subsubsection{Name and Year}
Victorian age, circa year 1830-1900.

\subsubsection{The Situation in English Society}
Queen Victoria ascended the English throne, and she promoted a society based on strict morals and social codes. Industrial, imperialism, peace (peace for the british population specifically) and great progress in many sciences lead to a massive rise in population. Many moved into the cities for jobs, but the early industrialism provided harsh working environments. The middle class grew and became concerned with social misery and injustice.

\subsubsection{Influence in Litterature}
In the Victorian era, a.k.a. the Golden Age of the English Novel, novels started being written by women and consumed by women. These novels were often focused on social interaction and lives, both of the rich and poor (Charles Dickens).

\subsubsection{Common Forms of Litterature}
The novel.

\subsubsection{Famous Authors and Works}
Oliver Twist \textit{(Charles Dickens)}, A Christmas Carol \textit{(Charles Dickens)}, Pride and Prejudice \textit{(Jane Austen)}.

\subsubsection{\textit{One Famous Work Example}}
Oliver Twist \textit{(Charles Dickens)}. It's about a boy starting life as a poor orphan.

\newpage
\subsection{Modern Times}

\subsubsection{Name and Year}
Modern Times, circa year 1900-now

\subsubsection{The Situation in English Society}
Two large wars and mass murder began and ended, feminism became a thing finally, and the overall living standards kept getting better.

\subsubsection{Influence in Litterature}
Literary history from this time and onwards is much more complex and diverse than before. Two large scale wars, the second one worse than the previous, affected the world. Evolution theory made an impact.

\subsubsection{Common Forms of Litterature}
Many genres popped up and became popular, such as postmodernism, social realism, science fiction, fantasy, feminism and crime.

\subsubsection{Famous Authors and Works}
Heart of Darkness \textit{(Joseph Conrad)}, Mrs. Dalloway \textit{(Virginia Woolf)}.

\subsubsection{\textit{One Famous Work Example}}
Heart of Darkness \textit{(Joseph Conrad)}.










