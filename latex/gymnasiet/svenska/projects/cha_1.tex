\subsection{Romantiken - Dracula}

\textit{Dracula är en vampyr, en annan art som dricker blod – kan man spåra referenser till främlingsfientlighet och sjukdomsskräck i berättelsen?} \bigskip

Dracula har starka kopplingar till sjukdomar och okunskap kring sjukdomar. De första referenserna till vampyrer som vi har är från ca år 1050 efter kristus. De fungerade som syndabockar för sjukdomar, precis som många andra mytiska och demoniska varelser, under en tid då bakterieteori inte fanns.

En möjlig sjukdom som visar stark länk till vampyren är rabies. Rabies finns och fanns världen över. Det sprids från djur till människor via just bett och kan leda till symptomer som att undvika ljus, förändrade sömnrytmer och ökad aggression. Detta är tydliga länkar till vampyrer. En annan sjukdom som också kan vara länkad är pellagra, som skapas av en brist på vitamin B3, och leder till dermatit, diarré, demens och död. En del får även en hög känslighet mot ljus vilket kan kopplas till vampyrer.

De två ovannämnda sjukdomarna var epidemier under det så kallade \textit{den stora vampyrepidemin} och under denna tid spred sig idén om vampyrer över europa. Man grävde upp gravar för att se efter tecken av 'vampyrisering' av kroppar, i rädsla av tron om den ökade mängden vampyrer under denna tid.

Sammanfattningsvis så är vampyren absolut länkad till sjukdomsskräck.

\newpage
\subsection{Modernismen - Frågor}

\subsubsection{Förklara Begreppen}
\begin{itemize}
	\item \textbf{Världkrig:} Världkrig är ett krig som inkluderar många länder, över stora delar av världen.
	\item \textbf{Psykoanalys:} Detta är pseudovetenskap, som inkluderar en mängd teorier och tekniker för att arbeta med det \textit{undermedvetna}.
	\item \textbf{Modernism:} Detta är en förändring i filosofi och konst som skedde under slutet av 1800-talet, i väst. Kunskap och teknik förbättrades snabbt. 
	\item \textbf{Dadaism:} Detta var en radikal förändring inom konst i början av 1900-talet. Realismen byttes ut mot nonsens.
	\item \textbf{Futurism:} Detta var en konstnärlig och social rörelse som lade vikt i teknologi.
	\item \textbf{Surrealism:} Här var det overkliga, viktigt.
\end{itemize}

\subsubsection{Kafka}
När Gregor Samsa vaknar en morgon ur sina oroliga drömmar ser han två svartklädda män i hans dörrkant. Gregor vet inte vem de är, men han välkomnar dem och undrar om de vill ha kaffe. De står kvar i dörrkanten utan svar och tittar på honom genom sina svarta glasögon. --15 MIN HAR GÅTT

\subsubsection{Förklara orden}
\begin{itemize}
	\item \textbf{Anbringa:} placera/fästa
	\item \textbf{Förödmjukande:} kränkande/sårande
	\item \textbf{Möda} ansträngning/besvär
	\item \textbf{Obetvingad} obesegrad
	\item \textbf{Prygel} pisk/aga
	\item \textbf{Oklanderligt} felfri
	\item \textbf{Eggen} vass kant/blad
\end{itemize}

\newpage
\subsection{Tema Dystopi}

\paragraph{Det är vad som väntar oss}

Dystopin har länge varit en populär genre, tyvärr på grund av världens funktion som inspirationskälla. Genrens litteratur är en spegling av omständigheterna i författarens samtid. Under tiden före och runt andra världskriget så skrevs det dystopier om totalitära övervakningssamhällen, vid kalla kriget så skiftade det istället till apokalypsscenarion som konsekvens till kärnvapen och runt år 1980 så började de ekologiska katastroferna falla i smaken.

Vårt nutid tror jag kan inspirera många typer av dystopier. Spänningar mellan västvärlden och Ryssland har återkommit som konsekvens av kriget i Ukraina, vilket lätt ger inspiration till scenarion runt kärnvapenkrig. Ett ökande och accelererande hot av klimatkris råder världen över vilket kan ge upphov till litteratur runt just det. Kvinnorättigheter plockas åter igen bort i världen med religion som motivation, vilket naturligtvis har fått starka motreaktioner som också speglas i litteratur. Vi har ingen brist på mardrömsscenarion att skriva om.

Boken som jag tror kan skrivas, är om ett scenario av krig, världskrig. Kärnvapen står på full beredskap, men ingen har avfyrat än. Den globala konflikten har satt klimatkrisen i låg prioritet, vilket har lett till massdöd i tredje världen som konsekvens av extremklimat under de många år som har pågått. Insekterna är sedan långt tillbaka något mycket sällsynt, och allt har fallit med dem. Matbrist råder världen över och de fattigaste drabbas hårdast, som med allt annat. Perspektivet är ur första person och vi får se världen genom ögonen av Matilda, fjorton år, från Tyskland. Vi får uppleva hennes sista 48 timmar, där klusterbomberna tär ner staden runt om henne. Hon gömmer sig i källaren av sin lägenhetsbyggnad med sin familj och andra från boendet. De sista timmarna spenderas i mörker och tystnad, fastklämd i ruinerna av hennes före detta lägenhetsbyggnad. Vilken person boken väljer att se genom har inte så stor betydelse, då allt för många ser sitt slut på olika men ändå liknande vis. Det är vad som väntar oss.





















































