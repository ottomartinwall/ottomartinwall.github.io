\subsection{Romantiken - Dracula}

\textit{Dracula är en vampyr, en annan art som dricker blod – kan man spåra referenser till främlingsfientlighet och sjukdomsskräck i berättelsen?} \bigskip

Dracula har starka kopplingar till sjukdomar och okunskap kring sjukdomar. De första referenserna till vampyrer som vi har är från ca år 1050 efter kristus. De fungerade som syndabockar för sjukdomar, precis som många andra mytiska och demoniska varelser, under en tid då bakterieteori inte fanns.

En möjlig sjukdom som visar stark länk till vampyren är rabies. Rabies finns och fanns världen över. Det sprids från djur till människor via just bett och kan leda till symptomer som att undvika ljus, förändrade sömnrytmer och ökad aggression. Detta är tydliga länkar till vampyrer. En annan sjukdom som också kan vara länkad är pellagra, som skapas av en brist på vitamin B3, och leder till dermatit, diarré, demens och död. En del får även en hög känslighet mot ljus vilket kan kopplas till vampyrer.

De två ovannämnda sjukdomarna var epidemier under det så kallade \textit{den stora vampyrepidemin} och under denna tid spred sig idén om vampyrer över europa. Man grävde upp gravar för att se efter tecken av 'vampyrisering' av kroppar, i rädsla av tron om den ökade mängden vampyrer under denna tid.

Sammanfattningsvis så är vampyren absolut länkad till sjukdomsskräck.

























