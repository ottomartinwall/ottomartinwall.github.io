\subsection{Den nya klassdebatten - PM}

\textbf{"Om klasstillhörigheten tidigare avgjordes av tillgången på pengar har i dag det kulturella och sociala kapitalet större betydelse" säger studenten Hanna Adenbäck. Hur påverkar klasskillnaderna livet för unga i Sverige?}

I en artikelserie från Dagens Nyheter vid namnet \textit{Den nya klassdebatten} så intervjuas Hanna Adenbäck. Hon menar att klasskillnader idag inte i stort sett uppstår från skillnader i ekonomi, men istället av kultur. Hon förklarar att trots att hennes tjänst är "lägst" och hon tjänar "minst" pengar så känner hon sig minst som arbetare bland hennes kollegor. Hon lyfter fram några exempel som att när arbetskamraterna läser i Veckorevyn så läser hon DN och att när de talar om träning och dieter så vill hon prata politik och konst. 

Adenbäck berättar om hur hon först märkte någon effekt av klasstillhörighet när hon började på konstskola. Där så har nästan ingen pengar då nästan alla är fattiga studenter, men hon märkte att andra klasskamrater till exempel kände till fler konstnärer och hade föräldrar som var själva var konstnärer, vilket hon påstod kan ha gett dem en större självkänsla. Hon menar då att hon kände en viss kulturell och social fattigdom.

Ett annat perspektiv är från Zlatan Ibrahimovic. I hans biografi, \textit{Jag är Zlatan Ibrahimovic}, så finns det en part där han skildrar sin början av gymnasietiden i Malmös Rosengård. Han berättar om hur där han tidigare gått så användes gympaskor och träningskläder, men att vid hans nya skola så var standarden istället Ralph Laurentröjor, Timberlandskor och skjortor. Han berättar om att han stod ut på grund av hans klädstil och att han inom kort tog sina enda pengar och köpte just den typen av kläder. I biografin så står det även om hur Zlatans kompis lyckades hänga ihop med en tjej som hade en bror med passande kläder, vilket han lånade. Ett citat från biografin beskriver den totala känslan väldigt väl: \textit{"Vi från förorten flöt aldrig in riktigt. Vi var annorlunda."}.

En tredje synvinkel kan vi få från antologin \textit{Skitliv}. Där beskrivs erfarenheter från unga arbetare, varav en heter My och är 23 år gammal. Hon arbetar på ett bemanningsföretag, vilket betyder att företaget hyr ut henne till andra företag. Effekten blir att hon inte får någon fast anställning på fulltid, vilket gör att inkomsten blir väldigt oförutsägbar. Hon fick under en tid en tillräcklig stabil inkomst för att flytta hemifrån men det slutade ganska tvärt. Hon väntade på att bemanningsföretaget skulle ringa henne för nästa uthyrning men inget kom, och sparkontot sinade. \textit{"Jag blev deprimerad och fick panikångest. Jag sov hela dagarna och kunde inte göra någonting. Jag hade ju inga pengar."} beskriver My. I boken beskrivs det att situationen som My hamnat i inte är ovanlig i bemanningsbranchen, där många unga arbetare hamnar.

Differenser i kultur och ekonomi skapar osäkerhet och utanförskap vilket träder fram i exempelupplevelserna hos Hannah, som upplevde en kulturell och social fattigdom, Zlatan, som spenderande sina sista pengar på en dyr skjorta och My utan jobb men med panikångest. Skillnaden mellan samhällsklasser har absolut en tydlig påverkan för unga idag.

\newpage
\subsection{Jämförande diktanalys}

\paragraph{Till Förruttnelsen}

Jag tror dikten handlar om en man som är påväg mot döden, och omfamnar dess ankomst. Giftermålet med bruden är en metafor för ett giftermål med döden. Bruden är döden.

“hasta, o älskade brud, att bädda vårt ensliga läger!”

Deras ensliga läger är kistan, under marken. Utifrån endast detta citat kan det vara svårt att låsa ned metaforen men strax under skrivs detta:

“Fort, smycka vår kammar -- på svartklädda båren den suckande älskarn din boning skall nå.”

Den svartklädda båren är en svart kista. Bår kan vara en dödsbädd, en kista. Den suckande älskarn, mannen, ska färdas till brudens boning, dödens hem, via sin kista.

“Till vällustens ljuva, förtrollande kvalm oss svartklädda brudsvenner följa.”
“Vår bröllopssång ringes av klockornas malm, och gröna gardiner oss dölja.”

brudsvenner syftar vanligtvis på några som följer paret till kyrkan, men de är nu svartklädda, vilket brukar vara färgstilen på begravningar. I beskrivningen “och gröna gardiner oss dölja” så ser jag det som att döden och mannen är under marken, i sin kista, med gröna gardiner av gräs mellan dem och luften ovan.

“Jag plågas häruppe, men lycklig jag bliver därnere hos dig”
“förkväv i ditt famntag min smärta!”
“När stormarna ute på världshavet råda, när fasor den blodade jorden bebo, när fejderna rasa, vi slumra dock båda i gyllene ro.”

Mannen har en längtan till den kommande döden, han välkomnar lugnet och obryddheten. Han välkomnar att släppa på smärta och bekymmer.

\paragraph{Döden tänkte jag mig så}

Jag ser det som att texten är ur perspektivet av någon som sett döden. Jag tror personen beskriver i efterhand hur han uppfattade döden innan han såg det med egna ögon, att “Döden tänkte jag mig så” - innan han själv upplevde det. Hans tidigare vy på döden blev något så oskyldigt i jämförelse till vad han fick se.

Jag uppfattar detta ur hur fint och oskyldigt han uppfattade döden, att det bara var att “Då tog han mig och satte mig i korgen och när jag somnat, började han gå.”, men direkt efter slutet av meningen så avslutar han dikten med “Döden tänkte jag mig så”. Han “tänkte” i preteritum, och nu har han en annan syn på döden som är långt mycket grövre.

\paragraph{Jämförelse - Differenser}

Båda dikter har en tydlig koppling till döden. Den ena är dock någon som omfamnar döden och den andra, tvärtom.

Den första dikten, Till förruttnelsen, ser jag som perspektivet ur en döende man som välkomnar lugnet och obryddheten som finns i döden. Dess språk är väldigt tydligt från 1800-talet med ett språk som kan vara svårt att förstå initialt, men det är väldigt beskrivande och livfullt.

Den andra dikten, Döden tänkte jag mig så, ser jag som att det är skrivet ur perspektivet av en man som sett döden och tappat sin oskyldiga uppfattning av den. Dikten skrevs på 1950-talet och är därför mycket mer läsbar för mig. Språket är inte använt på ett lika livfullt och dramatiskt vis, jag uppfattar det som mer monotont. I dikten Till förruttnelsen så används många utropstecken och jag tycker dikten har ett bättre flöde än Döden tänkte jag mig så, dikten offrar den korrekta grammatiska följden av ord för att få ett bättre flöde; 

“När stormarna ute på världshavet råda, när fasor den blodade jorden bebo, när fejderna rasa, vi slumra dock båda i gyllene ro.”.

\newpage
\subsection{Argumenterande tal}

\subsubsection{Behåll distansundervisningen (för de som vill!)}

Under de senaste två åren så har många elever i Sverige fått känna på hur det är med distansundervisning. Plötsligt så stod skolorna tomma, trafiken blev glesare, och Zooms aktiekurs hoppade på en raket. Blandade åsikter kom in som alltid vid förändring. Nu så är distansundevisningen över, men jag tycker att vi bör behålla den för de som vill.

Det första intrycket jag fick från min klass var positiv. Sovmorgonen blev längre och stressen lägre, delvis då ingen tid gick åt att ta sig fram och tillbaka till skolan. Det var skönt att inte behöva hänga i klassrummen och istället så satt alla och pratade med varann i en egenskapad klasschatt på Discord.

Något som trädde in efter en tid var att många halkade efter i skolan. Gymnasiet, trots dess inriktningar, erbjuder ändå bara väldigt generella utbildningar, vilket resulterar i att för de allra flesta så är det bara mycket få ämnen man faktiskt tycker om. Den försämrade kommunikationen med lärare, som oundvikligt sker vid distansundervisning, i kombination med det nämnda ointresset, tror jag ledde denna denna situation.

En sak som är otroligt viktigt vid effektiv kommunikation, som ofta inte uppmärksammas, är det icke-verbala. Kommunikationsforskare brukar hävda att just det icke-verbala är av störst betydelse. När läraren står framme vid tavlan och resten av klassen sitter ner, så dras ögonen och uppmärksamheten mycket lättare mot materialet som visas, än om läraren bara är en av 25 ansikten på ens datorskärm.

Nu kanske ni undrar, \textit{"skulle inte du argumentera för att behålla distansundervisingen?"}. Till det svarar jag då, \textit{"absolut"}. Mångas skolresultat och motivation påverkades negativt, men jag skulle vilja påstå att för en betydelsefullt stor minoritet så var detta en väldigt positiv förändring. Jag tycker att denna period avslöjade för mig om hur lite arbete som egentligen behöver göras i gymnasiet. Väldigt mycket tid i skolan går vanligtvis åt att lyssna på föreläsningar som inte behövs, det går åt till att stå och vänta på starten av lektioner, det går åt till att gå mellan klassrum, och det går åt till att ta sig till skolan fram och tillbaka. Efter att många av dessa ineffektiviteter togs bort så upplevde jag att av all tid som innan spenderades på skolan, så var det faktiska arbetet bara en bråkdel av det. Plötsligt så hade jag möjligheten till en enorm fritid. Många av mina vänner började kunna spendera mer tid på gymmet och andra intressen. Många fick sova ut längre. Det fanns alltså mer tid till jaget, vilket är något väldigt viktigt för ens mentala hälsa.

Alla människor är olika och behöver olika omständigheter för att nå sin potential, och därmed tycker jag tycker att valet bör finnas till att få arbeta på distans.



















































