\subsection{Den nya klassdebatten - PM}

\textbf{"Om klasstillhörigheten tidigare avgjordes av tillgången på pengar har i dag det kulturella och sociala kapitalet större betydelse" säger studenten Hanna Adenbäck. Hur påverkar klasskillnaderna livet för unga i Sverige?}

I en artikelserie från Dagens Nyheter vid namnet \textit{Den nya klassdebatten} så intervjuas Hanna Adenbäck. Hon menar att klasskillnader idag inte i stort sett uppstår från skillnader i ekonomi, men istället av kultur. Hon förklarar att trots att hennes tjänst är "lägst" och hon tjänar "minst" pengar så känner hon sig minst som arbetare bland hennes kollegor. Hon lyfter fram några exempel som att när arbetskamraterna läser i Veckorevyn så läser hon DN och att när de talar om träning och dieter så vill hon prata politik och konst. 

Adenbäck berättar om hur hon först märkte någon effekt av klasstillhörighet när hon började på konstskola. Där så har nästan ingen pengar då nästan alla är fattiga studenter, men hon märkte att andra klasskamrater till exempel kände till fler konstnärer och hade föräldrar som var själva var konstnärer, vilket hon påstod kan ha gett dem en större självkänsla. Hon menar då att hon kände en viss kulturell och social fattigdom.

Ett annat perspektiv är från Zlatan Ibrahimovic. I hans biografi, \textit{Jag är Zlatan Ibrahimovic}, så finns det en part där han skildrar sin början av gymnasietiden i Malmös Rosengård. Han berättar om hur där han tidigare gått så användes gympaskor och träningskläder, men att vid hans nya skola så var standarden istället Ralph Laurentröjor, Timberlandskor och skjortor. Han berättar om att han stod ut på grund av hans klädstil och att han inom kort tog sina enda pengar och köpte just den typen av kläder. I biografin så står det även om hur Zlatans kompis lyckades hänga ihop med en tjej som hade en bror med passande kläder, vilket han lånade. Ett citat från biografin beskriver den totala känslan väldigt väl: \textit{"Vi från förorten flöt aldrig in riktigt. Vi var annorlunda."}.

En tredje synvinkel kan vi få från antologin \textit{Skitliv}. Där beskrivs erfarenheter från unga arbetare, varav en heter My och är 23 år gammal. Hon arbetar på ett bemanningsföretag, vilket betyder att företaget hyr ut henne till andra företag. Effekten blir att hon inte får någon fast anställning på fulltid, vilket gör att inkomsten blir väldigt oförutsägbar. Hon fick under en tid en tillräcklig stabil inkomst för att flytta hemifrån men det slutade ganska tvärt. Hon väntade på att bemanningsföretaget skulle ringa henne för nästa uthyrning men inget kom, och sparkontot sinade. \textit{"Jag blev deprimerad och fick panikångest. Jag sov hela dagarna och kunde inte göra någonting. Jag hade ju inga pengar."} beskriver My. I boken beskrivs det att situationen som My hamnat i inte är ovanlig i bemanningsbranchen, där många unga arbetare hamnar.

Differenser i kultur och ekonomi skapar osäkerhet och utanförskap vilket träder fram i exempelupplevelserna hos Hannah, som upplevde en kulturell och social fattigdom, Zlatan, som spenderande sina sista pengar på en dyr skjorta och My utan jobb men med panikångest. Skillnaden mellan samhällsklasser har absolut en tydlig påverkan för unga idag.






