\subsection{Den nya klassdebatten - PM}

\textbf{"Om klasstillhörigheten tidigare avgjordes av tillgången på pengar har i dag det kulturella och sociala kapitalet större betydelse" säger studenten Hanna Adenbäck. Hur påverkar klasskillnaderna livet för unga i Sverige?}

I en artikel från \textit{Dagens Nyheter} så intervjuas Hanna Adenbäck. Hon menar att klasskillnader idag inte i stort sett uppstår från skillnader i ekonomi, men istället av kultur. Hon förklarar att trots att hennes tjänst är "lägst" och hon tjänar "minst" pengar så känner hon sig minst som arbetare bland hennes kollegor. Hon lyfter fram några exempel som att när arbetskamraterna läser i Veckorevyn så läser hon DN och att när de talar om träning och dieter så vill hon prata politik och konst. 

Adenbäck berättar om hur hon först märkte en klasstillhörighet när hon började på konstskola. Där så har nästan ingen pengar då nästan alla är fattiga studenter, men hon märkte att andra klasskamrater till exempel kände till fler konstnärer och hade föräldrar som var själva var konstnärer, vilket hon påstod kan ha gett dem en större självkänsla. Hon menar då att hon kände en viss kulturell och social fattigdom. Adenbäck påpekar dock att pengar ger en möjligheten att förverkliga sina drömmar då det ger en person friheten att välja sin vardag.

Ett annat perspektiv är från Zlatan Ibrahimovic. I hans biografi så finns det en part där han skildrar sin början av sin gymnasietid i Malmös Rosengård. Han berättar om hur 

