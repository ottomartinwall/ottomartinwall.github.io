\subsection{Konjugat och kvadreringsreglerna}

Detta är bara att memorera. \bigskip

\begin{theorem}{Konjugatregeln}
	\begin{align}
	(a+b)(a-b) = a^2-b^2
	\end{align}
\end{theorem}

Motivering:

\begin{align}
	(a+b)(a-b) = \\ 
	a^2 - ab + ab + b^2 = \\
	a^2 + b^2
\end{align}

\begin{theorem}{Kvadreringsreglerna}
	\begin{align}
		(a+b)^2 = a^2+2ab+b^2 \\
		(a-b)^2 = a^2-2ab+b^2
	\end{align}
\end{theorem}

Motivering för första:

\begin{align}
	(a+b)^2 = \\
	(a+b)(a+b) = \\
	a^2 + ab + ba + b^2 = \\
	a^2 + 2ab + b^2
\end{align}

Motivering för andra:

\begin{align}
	(a-b)^2 = \\
	(a-b)(a-b) = \\
	a^2 - ab - ba + (-b)^2 = \\
	a^2 - 2ab + b^2
\end{align}

\newpage
\subsection{Polynom}

\begin{definition}
	En polynom är en funktion som kan skrivas i formen
	\begin{align}
		f(x) = a_0x^0 + a_1x^1 + a_2x^2 + \hdots + a_nx^n
	\end{align}
	där $n$ är ett positivt heltal från $0$ och uppåt och $a_0, a_1, a_2, \hdots, a_n$ är konstanter.
\end{definition}

Utifrån detta så kan vi se att linjära funktioner är polynomer där konstanten med index 1 inte är lika med noll, men att alla konstanter med högre index är lika med noll. Något som Matematik 2B kör hårt på är dock just andragradsfunktioner, vilket är polynomer där konstanten med index 2 inte är lika med noll men alla konstanter med högre index är lika med noll.

En andragradsfunktion kan därmed skrivas i formen

\begin{align}
	f(x) = ax^2 + bx + c
\end{align}

som det görs i de allra flesta matteböcker.

\subsubsection{}

\newpage
\subsection{Exponentialfunktion}

\begin{definition}
	En exponentialfunktion kan skrivas i formen
	\begin{align}
		f(x)=Ca^x
	\end{align}
	där $C$ och $a$ är konstanter.
 \end{definition}
 

Ett kännetecken av exponentialfunktioner är att när de ritas up grafiskt, och $a > 1$, så sticker grafen snabbt iväg uppåt i en allt högre takt. Då $a < 1$ dyker grafen neråt i en allt långsammare takt. Om $a=1$ så är funktionen bara konstant då $1^x=1$.

\subsubsection{Exponentialmodeller}

Exponentialfunktioner är bra på att beskriva scenarion där förändringshastigheten är relaterad till det nuvarande värdet. Det kan vara ett bankkonto med ränta som resulterar i att det växer allt snabbare varje år, eller en bakteriekultur som växer i en takt som beror på hur många bakterier det är vid varje tillfälle.

Vi kan testa skapa en modell som beskriver ett bankkonto med ränta. Vi låter $f(x)$ benämna bankkontots innehåll i kr, och $x$ vara tiden i år.

För att skapa en modell som beskriver detta scenario så väljer vi att $f(x)$ är benämner mängden pengar vid varje $x$, som representerar tid i år. Sedan väljer vi att starta vid 10000kr vid år 0.
\begin{align}
	f(x)=Ca^x \\
	f(0)=Ca^0 = C
\end{align}
Därmed är $C = 10000$. Då räntan är 2\% per år, så är $a=1.02$. Detta resulterar i att funktionen som beskriver scenariot är
\begin{align}
	f(x)=10000\cdot1.02^x.
\end{align}

\newpage
\subsection{Logaritm}

Logaritmen är inversfunktionen\footnote{Checka \ref{Inversfunktion} för förklaring av inversfunktion.} till exponentialen. Då exponentialfunktioner kan ha olika bas\footnote{bas är nedre delen av en potens, i $a^x$ är $a$ basen.}, så måste vi även ange den basen för logaritmen så att logaritmfunktionen faktiskt blir inversen. I matematik 2B så används bara basen 10 för logaritmen.

Här under är $g(x)$ inversen av $f(x)$, och $\log_{10}$ benämner att logaritmen använder basen 10.
\begin{align}
	f(x) = 10^x \\
	g(x) = \log_{10}(x)
\end{align}
Då $g(x)$ är inversfunktionen till $f(x)$ så gäller detta:
\begin{align}
	f(g(x)) = 10^{g(x)} = 10^{\log_{10}(x)} = x.
\end{align}
Detta leder till att
\begin{align}
	\log_{10}(10) = 1 \\
	\log_{10}(100) = 2 \\
	\log_{10}(1000) = 3 \\
	\text{o.s.v.}
\end{align}
vilket är användbart att komma ihåg.

\newpage
\subsubsection{Logaritmlagar}

Här använder jag förkortningen $\lg(x) = \log_{10}(x)$, då vi endast använder logaritmer med bas 10 i matematik 2B. Du kan också använda den förkortningen.

\begin{align}
	10^{\lg(a)}=a \\
	\lg(a^b) = \lg(a)\cdot b \\
	\lg(a)+\lg(b) = \lg(ab) \\
	\lg(a)-\lg(b) = \frac{\lg(a)}{\lg(b)} \\
\end{align} 






































































