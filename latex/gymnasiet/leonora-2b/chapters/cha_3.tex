\subsection{Konjugat och kvadreringsreglerna}

Detta är bara att memorera. \bigskip

\begin{theorem}{Konjugatregeln}
	\begin{align}
	(a+b)(a-b) = a^2-b^2
	\end{align}
\end{theorem}

Motivering:

\begin{align}
	(a+b)(a-b) = \\ 
	a^2 - ab + ab + b^2 = \\
	a^2 + b^2
\end{align}

\begin{theorem}{Kvadreringsreglerna}
	\begin{align}
		(a+b)^2 = a^2+2ab+b^2 \\
		(a-b)^2 = a^2-2ab+b^2
	\end{align}
\end{theorem}

Motivering för första:

\begin{align}
	(a+b)^2 = \\
	(a+b)(a+b) = \\
	a^2 + ab + ba + b^2 = \\
	a^2 + 2ab + b^2
\end{align}

Motivering för andra:

\begin{align}
	(a-b)^2 = \\
	(a-b)(a-b) = \\
	a^2 - ab - ba + (-b)^2 = \\
	a^2 - 2ab + b^2
\end{align}

\newpage
\subsection{Polynom}

\begin{definition}
	En polynom är en funktion som kan skrivas i formen
	
	\begin{align}
		f(x) = a_0x^0 + a_1x^1 + a_2x^2 + \hdots + a_nx^n
	\end{align}
	
	där $n$ är ett positivt heltal från $0$ och uppåt och $a_0, a_1, a_2, \hdots, a_n$ är konstanter.
\end{definition}

Utifrån detta så kan vi se att linjära funktioner är polynomer där konstanten med index 1 inte är lika med noll, men att alla konstanter med högre index är lika med noll. Något som Matematik 2B kör hårt på är dock just andragradsfunktioner, vilket är polynomer där konstanten med index 2 inte är lika med noll men alla konstanter med högre index är lika med noll.

En andragradsfunktion kan därmed skrivas i formen

\begin{align}
	f(x) = ax^2 + bx + c
\end{align}

som det görs i de allra flesta matteböcker.








































































