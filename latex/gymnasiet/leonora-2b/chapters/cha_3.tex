\subsection{Konjugat och kvadreringsreglerna}

Detta är bara att memorera. \bigskip

\begin{theorem}{Konjugatregeln}
	\begin{align}
	(a+b)(a-b) = a^2-b^2
	\end{align}
\end{theorem}

Motivering:

\begin{align}
	(a+b)(a-b) = \\ 
	a^2 - ab + ab + b^2 = \\
	a^2 + b^2
\end{align}

\begin{theorem}{Kvadreringsreglerna}
	\begin{align}
		(a+b)^2 = a^2+2ab+b^2 \\
		(a-b)^2 = a^2-2ab+b^2
	\end{align}
\end{theorem}

Motivering för första:

\begin{align}
	(a+b)^2 = \\
	(a+b)(a+b) = \\
	a^2 + ab + ba + b^2 = \\
	a^2 + 2ab + b^2
\end{align}

Motivering för andra:

\begin{align}
	(a-b)^2 = \\
	(a-b)(a-b) = \\
	a^2 - ab - ba + (-b)^2 = \\
	a^2 - 2ab + b^2
\end{align}

\newpage
\subsection{Polynom}

\begin{definition}
	En polynom är en funktion som kan skrivas i formen
	\begin{align}
		f(x) = a_0x^0 + a_1x^1 + a_2x^2 + \hdots + a_nx^n
	\end{align}
	där $n$ är ett positivt heltal från $0$ och uppåt och $a_0, a_1, a_2, \hdots, a_n$ är konstanter.
\end{definition}

Utifrån detta så kan vi se att linjära funktioner är polynomer där konstanten med index 1 inte är lika med noll, men att alla konstanter med högre index är lika med noll. Något som Matematik 2B kör hårt på är dock just andragradsfunktioner, vilket är polynomer där konstanten med index 2 inte är lika med noll men alla konstanter med högre index är lika med noll.

En andragradsfunktion kan därmed skrivas i formen

\begin{align}
	f(x) = ax^2 + bx + c
\end{align}

som det görs i de allra flesta matteböcker.

\newpage
\subsection{Exponentialfunktion}

\begin{definition}
	En exponentialfunktion kan skrivas i formen
	\begin{align}
		f(x)=Ca^x
	\end{align}
	där $C$ och $a$ är konstanter.
 \end{definition}
 

Ett kännetecken av exponentialfunktioner är att när de ritas up grafiskt, och $a > 1$, så sticker grafen snabbt iväg uppåt i en allt högre takt. Då $a < 1$ dyker grafen neråt i en allt långsammare takt. Om $a=1$ så är funktionen bara konstant då $1^x=1$.

\subsubsection{Exponentialmodeller}

Exponentialfunktioner är bra på att beskriva saker där något växer eller minskar i en hastighet som är kopplad till det värde det är på vid tillfället. Ett exempel är ett bankkonto med ränta, som då växer olika mängd i faktiska kronor beroende på hur stort kontot är. Om/när kontot har 100 000kr så resulterar en 2\% ränta i att du får 2000kr per år, och om/när kontot har 1 000 000kr så resulterar det i 20 000kr.

För att skapa en modell som beskriver detta scenario så väljer vi att $f(x)$ är benämner mängden pengar vid varje $x$, som representerar tid i år. Sedan väljer vi att starta vid 10000kr vid år 0.
\begin{align}
	f(x)=Ca^x \\
	f(0)=Ca^0 = C
\end{align}
Därmed är $C = 10000$. Då räntan är 2\% per år, så är $a=1.02$. Detta resulterar i att funktionen som beskriver scenariot är
\begin{align}
	f(x)=10000\cdot1.02^x.
\end{align}

\newpage
\subsection{Logaritm}

Logaritmen är inversfunktionen\footnote{Checka \ref{Inversfunktion} för förklaring av inversfunktion.} till exponentialen.




































































