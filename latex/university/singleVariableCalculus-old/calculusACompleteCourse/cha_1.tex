\subsection{Limits and Continuity}

\begin{theorem}[The Squeeze Theorem, 4]
	Suppose that $f(x) \leq g(x) \leq h(x)$ holds for all $x$ in some open interval containing $c$, except possibly at $x=c$. Suppose also that
	
	\begin{align*}
		\lim_{x \rightarrow c} f(x) = \lim_{x \rightarrow c} h(x) = L
	\end{align*}
	
	Then $\lim_{x \rightarrow c} g(x) = L $.
\end{theorem}

\begin{proof}
	For this proof, the $(\epsilon, \delta)$-definition of the limit will be used.
	
	The goal is to prove that $\lim_{x \rightarrow c} g(x) = L$, which is true if
	
	\begin{align*}
		\forall \epsilon > 0, \exists \delta > 0 : \forall x, (|x-c| < \delta \Rightarrow |g(x)-L| < \epsilon).
	\end{align*}
	
	Since $\lim_{x \rightarrow c} f(x) = L$,
	
	\begin{align}
		\forall \epsilon > 0, \exists \delta_1 > 0 : \forall x, (|x-c| < \delta_1 \Rightarrow |f(x)-L| < \epsilon)
	\end{align}
	
	And since $\lim_{x \rightarrow c} h(x) = L$,
	
	\begin{align}
		\forall \epsilon > 0, \exists \delta_2 > 0 : \forall x, (|x-c| < \delta_2 \Rightarrow |h(x)-L| < \epsilon).
	\end{align}
	
	Then we have 
	
	\begin{align*}
		&f(x) \leq g(x) \leq h(x) \\
		&f(x)-L \leq g(x)-L \leq h(x)-L
	\end{align*}
	
	We can choose $\delta = \min\{\delta_1, \delta_2\}$, then if $|x-c| < \delta$, and combining (1) and (2), we have
	
	\begin{align*}
		&-\epsilon < f(x)-L \leq g(x)-L \leq h(x)-L < \epsilon \\
		&-\epsilon < g(x)-L < \epsilon \\
		&|g(x)-L| < \epsilon
	\end{align*}
	
	So $\lim_{x \rightarrow c} g(x) = L$, which completes the proof.
\end{proof}

\begin{theorem}[The Intermediate-Value Theorem, 9]
	If $f(x)$ is continuous on the interval $[a, b]$ and if $s$ is a number between $f(a)$ and $f(b)$, then there exists a number c in $[a, b]$ such that $f(c) = s$.
	
	In particular, a continuous function defined on a closed interval takes on all values between its minimum value m and its maximum value M , so its range is also a closed interval, $[m, M]$.
\end{theorem}

\begin{proof}
	
\end{proof}

\newpage

\subsubsection{Exercises 1.1}
\subsubsection{Exercises 1.2}

\paragraph{78. What is the domain of $\sin \frac{1}{x}$ ? Evaluate $\lim_{x\rightarrow 0} x\sin \frac{1}{x}$.}

The domain of $x\sin x$ is $\mathbb{R}$. The domain of $\frac{1}{x}$ is $(-\infty, 0)\cup(0, \infty)$. Therefore, the domain of $x\sin \frac{1}{x}$ is $(-\infty, 0)\cup(0, \infty)$.

To evaluate $\lim_{x\rightarrow 0} x\sin \frac{1}{x}$, we can first evaluate $\lim_{x\rightarrow 0} \frac{1}{x}$. 

$\lim_{x\rightarrow 0^+} \frac{1}{x} = +\infty$, $\lim_{x\rightarrow 0^-} \frac{1}{x} = -\infty$. 

This means that $\lim_{x\rightarrow 0} \sin \frac{1}{x} = \lim_{x\rightarrow \pm\infty} \sin x$, which means that $-1 \leq \lim_{x\rightarrow 0} \sin \frac{1}{x} \leq 1$.

$\lim_{x\rightarrow 0} x\sin \frac{1}{x} = (\lim_{x\rightarrow 0} x)(\lim_{x\rightarrow 0} \sin \frac{1}{x}) = 0$ 

\paragraph{79. Suppose $|f(x)| \leq g(x) \forall x$. What can you conclude about $\lim_{x\rightarrow a} f(x)$ if $\lim_{x\rightarrow a} g(x) = 0$ ? What if $\lim_{x\rightarrow a} g(x) = 3$ ?}

$|f(x)| \leq g(x) \forall x \Leftrightarrow -g(x) \leq f(x) \leq g(x) \forall x$. Since $\lim_{x\rightarrow a} g(x) = 0$ and therefore $\lim_{x\rightarrow a} -g(x) = 0$, then $\lim_{x\rightarrow a} f(x) = 0$ by the squeeze theorem.

If $\lim_{x\rightarrow a} g(x) = 3$, and $-g(x) \leq f(x) \leq g(x) \forall x$, then we can conclude that either $-3 \leq \lim_{x\rightarrow a} f(x) \leq 3$, or $\lim_{x\rightarrow a} f(x)$ doesn't exist.

\subsubsection{Exercises 1.3}

\subsubsection{Exercises 1.4}

\paragraph{32.}

Let $g(x) = f(x) - x$. Since $0 \leq f(x) \leq 1$ for $0 \leq x \leq 1$, then $0 \leq g(0)$. By the same argument, $g(1) \leq 0$. Because $g(x)$ is continuous in the interval $[0,1]$, there must be some value $c \in [0,1]$ such that $g(c)=0$, by the Intermediate-Value Theorem. If $g(c)=0$, then $f(c)=c$, which was to be shown.

\paragraph{33.}

Since $f(x)$ is even, it is symmetric around the y-axis. The symmetric equivilance of $\lim_{x\rightarrow 0^+}$ around the y-axis is $\lim_{x\rightarrow 0^-}$. Since $f(x)$ is right-continuous, it means that $\lim_{x\rightarrow 0+} f(x) = f(0)$ and because of the symmetry, $\lim_{x\rightarrow 0-} f(x) = f(0)$. Because $\lim_{x\rightarrow 0+} f(x) = \lim_{x\rightarrow 0-} f(x) = f(0)$, $f$ is continuous at $x=0$.

\paragraph{34.}

$\lim_{x\rightarrow 0^+} f(x) = f(0)$ because $f$ is right continuous. Since $f$ is odd, it is symmetric around the origin, and therefore $\lim_{x\rightarrow 0^-} f(x) = \lim_{x\rightarrow 0^+} f(x) = f(0) = 0$. Since $f$ is both right and left continuous at $x=0$, it is continuous at $x=0$.

\newpage
\subsubsection{Exercises 1.5}

\paragraph{31.}

$(\lim_{x\rightarrow a} f(x) = L) \Leftrightarrow (\forall \epsilon > 0, \exists \delta_1 > 0 : 0<|x-a|<\delta_1 \Rightarrow |f(x)-L|<\epsilon)$

and 

$(\lim_{x\rightarrow a} f(x) = M) \Leftrightarrow (\forall \epsilon > 0, \exists \delta_2 > 0 : 0<|x-a|<\delta_2 \Rightarrow |f(x)-M|<\epsilon)$

We assume that $L \neq M$. If we choose $\delta = \min\{\delta_1, \delta_2\}$, then $0<|x-a|<\delta \Rightarrow |f(x)-L| + |f(x)-M| < \epsilon + \epsilon = 2\epsilon$. 

By the triangle inequality,

$|f(x)-L| + |f(x)-M| \geq |(L-f(x)) + (f(x)-M)| = |L-M|$.

Since $L \neq M$, we can let $\epsilon = |L-M|/4$ because $|L-M|$ is positive.

This means that $|L-M| \leq 2\epsilon = |L-M|/2 \Rightarrow 2 \leq 1$, which is obviously false. Therefore $L = M$ and the limit is unique, which was to be shown.

\newpage
\paragraph{32.}

Since $\lim_{x\rightarrow a} g(x) = M$, the following must be true

\begin{align*}
	|g(x)| = |(g(x)-M)+M| \leq |g(x)-M|+|M| < \epsilon + |M|
\end{align*}

If we choose $\epsilon = 1$ then

\begin{align*}
	|g(x)| < 1 + |M|
\end{align*}

Which was to be shown.

\paragraph{33.}

$\lim_{x\rightarrow a} f(x) \Leftrightarrow (\forall \epsilon > 0, \exists \delta_1 > 0 : \forall x, (0<|x-a|<\delta_1 \Rightarrow |f(x)-L|<\epsilon))$

And

$\lim_{x\rightarrow a} g(x) \Leftrightarrow (\forall \epsilon > 0, \exists \delta_2 > 0 : \forall x, (0<|x-a|<\delta_2 \Rightarrow |g(x)-M|<\epsilon))$

Lets assume that $\lim_{x\rightarrow a} f(x)g(x) \neq LM$.

Let $\delta = \min\{\delta_1, \delta_2\}$. This would result in that

\begin{align}
	\forall \epsilon > 0, \exists \delta > 0 : \forall x, (0<|x-a|<\delta \Rightarrow |f(x)-L| + |g(x)-M| < 2\epsilon)
\end{align}

$|f(x)-L| + |g(x)-M| \geq |g(x)||f(x)-L| + |L||g(x)-M| = |g(x)(f(x)-L)| + |L(g(x)-M)| \geq |g(x)(f(x)-L) + L(g(x)-M)| = |f(x)g(x) - Lg(x)+Lg(x) - LM| = |(f(x)g(x)) - (LM)|$

This together with (3) means that

$\forall \epsilon > 0, \exists \delta > 0 : \forall x, (0<|x-a|<\delta \Rightarrow |(f(x)g(x)) - (LM)| < 2\epsilon)$

Which shows that lim $\lim_{x\rightarrow a} f(x)g(x) = LM$.
























































