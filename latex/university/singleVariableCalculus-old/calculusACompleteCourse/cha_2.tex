\subsection{Tangent Lines and Their Slopes}

\subsubsection{Exercises 2.1}

\paragraph{32.}

\begin{align*}
	P(x) &= a_0 + a_1(x-a) + a_2(x-a)^2 + \hdots + a_n(x-a)^n \\
	P^{(n)}(a) &= n!a_n 
\end{align*}

This means that $P'(a)=a_1$.

\begin{align*}
	l(x) &= m(x-a)+b \\
	l'(a) &= m
\end{align*}

\begin{align*}
	P(x)-l(x) &= (x-a)^2Q(x) \\
	P(x) &= l(x) + (x-a)^2Q(x) \\
	P(x) &= b + m(x-a) + (x-a)^2Q(x) \\
	P'(x) &= m + 2(x-a)Q(x) + (x-a)^2Q'(x) \\
	P'(a) &= m
\end{align*}

Since $P'(a)=a_1=m=l'(a)$, $P$ and $l$ has the same tangent in $x=a$.

\subsubsection{Exercises 2.2}

\paragraph{52.}

\begin{align*}
	a^n-b^n = (a-b)(a^{n-1}+a^{n-2}b+a^{n-3}b^2+\hdots+ab^{n-2}+b^{n-1})
\end{align*}


\begin{align*}
	\frac{d}{dx}x^{-n} = \lim_{h\rightarrow 0} \frac{(x+h)^{-n}-x^{-n}}{h} = \\
	= \lim_{h\rightarrow 0} \left(\frac{1}{h(x+h)^{-n}}+\frac{1}{hx^{-n}}\right) = \\
\end{align*}


