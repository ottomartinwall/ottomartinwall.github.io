\subsection{Limits and Continuity}

\begin{theorem}[The Squeeze Theorem]
	Suppose that $f(x) \leq g(x) \leq h(x)$ holds for all $x$ in some open interval containing $c$, except possibly at $x=c$. Suppose also that
	
	\begin{align*}
		\lim_{x \rightarrow c} f(x) = \lim_{x \rightarrow c} h(x) = L
	\end{align*}
	
	Then $\lim_{x \rightarrow c} g(x) = L $.
\end{theorem}

\begin{proof}
	For this proof, the $(\epsilon, \delta)$-definition of the limit will be used.
	
	The goal is to prove that $\lim_{x \rightarrow c} g(x) = L$, which is true if
	
	\begin{align*}
		\forall \epsilon > 0, \exists \delta > 0 : \forall x, (|x-a| < \delta \Rightarrow |g(x)-L| < \epsilon).
	\end{align*}
	
	Since $\lim_{x \rightarrow c} f(x) = L$,
	
	\begin{align}
		\forall \epsilon > 0, \exists \delta_1 > 0 : \forall x, (|x-a| < \delta_1 \Rightarrow |f(x)-L| < \epsilon)
	\end{align}
	
	And since $\lim_{x \rightarrow c} h(x) = L$,
	
	\begin{align}
		\forall \epsilon > 0, \exists \delta_2 > 0 : \forall x, (|x-a| < \delta_2 \Rightarrow |h(x)-L| < \epsilon).
	\end{align}
	
	Then we have 
	
	\begin{align*}
		&f(x) \leq g(x) \leq h(x) \\
		&f(x)-L \leq g(x)-L \leq h(x)-L
	\end{align*}
	
	We can choose $\delta = \min\{\delta_1, \delta_2\}$, then if $|x-a| < \delta$, and combining (1) and (2), we have
	
	\begin{align*}
		&-\epsilon < f(x)-L \leq g(x)-L \leq h(x)-L < \epsilon \\
		&-\epsilon < g(x)-L < \epsilon \\
		&|g(x)-L| < \epsilon
	\end{align*}
	
	So $\lim_{x \rightarrow c} g(x) = L$, which completes the proof.
\end{proof}
\newpage

\subsubsection{Exercises 1.1}
\subsubsection{Exercises 1.2}









