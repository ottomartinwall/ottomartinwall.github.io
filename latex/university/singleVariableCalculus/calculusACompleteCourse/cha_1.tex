\subsection{Limits and Continuity}

\begin{theorem}[The Squeeze Theorem, 4]
	Suppose that $f(x) \leq g(x) \leq h(x)$ holds for all $x$ in some open interval containing $c$, except possibly at $x=c$. Suppose also that
	
	\begin{align*}
		\lim_{x \rightarrow c} f(x) = \lim_{x \rightarrow c} h(x) = L
	\end{align*}
	
	Then $\lim_{x \rightarrow c} g(x) = L $.
\end{theorem}

\begin{proof}
	For this proof, the $(\epsilon, \delta)$-definition of the limit will be used.
	
	The goal is to prove that $\lim_{x \rightarrow c} g(x) = L$, which is true if
	
	\begin{align*}
		\forall \epsilon > 0, \exists \delta > 0 : \forall x, (|x-c| < \delta \Rightarrow |g(x)-L| < \epsilon).
	\end{align*}
	
	Since $\lim_{x \rightarrow c} f(x) = L$,
	
	\begin{align}
		\forall \epsilon > 0, \exists \delta_1 > 0 : \forall x, (|x-c| < \delta_1 \Rightarrow |f(x)-L| < \epsilon)
	\end{align}
	
	And since $\lim_{x \rightarrow c} h(x) = L$,
	
	\begin{align}
		\forall \epsilon > 0, \exists \delta_2 > 0 : \forall x, (|x-c| < \delta_2 \Rightarrow |h(x)-L| < \epsilon).
	\end{align}
	
	Then we have 
	
	\begin{align*}
		&f(x) \leq g(x) \leq h(x) \\
		&f(x)-L \leq g(x)-L \leq h(x)-L
	\end{align*}
	
	We can choose $\delta = \min\{\delta_1, \delta_2\}$, then if $|x-c| < \delta$, and combining (1) and (2), we have
	
	\begin{align*}
		&-\epsilon < f(x)-L \leq g(x)-L \leq h(x)-L < \epsilon \\
		&-\epsilon < g(x)-L < \epsilon \\
		&|g(x)-L| < \epsilon
	\end{align*}
	
	So $\lim_{x \rightarrow c} g(x) = L$, which completes the proof.
\end{proof}
\newpage

\subsubsection{Exercises 1.1}
\subsubsection{Exercises 1.2}

\paragraph{78. What is the domain of $\sin \frac{1}{x}$ ? Evaluate $\lim_{x\rightarrow 0} x\sin \frac{1}{x}$.}

The domain of $x\sin x$ is $\mathbb{R}$. The domain of $\frac{1}{x}$ is $(-\infty, 0)\cup(0, \infty)$. Therefore, the domain of $x\sin \frac{1}{x}$ is $(-\infty, 0)\cup(0, \infty)$.

To evaluate $\lim_{x\rightarrow 0} x\sin \frac{1}{x}$, we can first evaluate $\lim_{x\rightarrow 0} \frac{1}{x}$. 

$\lim_{x\rightarrow 0^+} \frac{1}{x} = +\infty$, $\lim_{x\rightarrow 0^-} \frac{1}{x} = -\infty$. 

This means that $\lim_{x\rightarrow 0} \sin \frac{1}{x} = \lim_{x\rightarrow \pm\infty} \sin x$, which means that $-1 \leq \lim_{x\rightarrow 0} \sin \frac{1}{x} \leq 1$.

$\lim_{x\rightarrow 0} x\sin \frac{1}{x} = (\lim_{x\rightarrow 0} x)(\lim_{x\rightarrow 0} \sin \frac{1}{x}) = 0$ 

\paragraph{79. Suppose $|f(x)| \leq g(x) \forall x$. What can you conclude about $\lim_{x\rightarrow a} f(x)$ if $\lim_{x\rightarrow a} g(x) = 0$ ? What if $\lim_{x\rightarrow a} g(x) = 3$ ?}

$|f(x)| \leq g(x) \forall x \Leftrightarrow -g(x) \leq f(x) \leq g(x) \forall x$. Since $\lim_{x\rightarrow a} g(x) = 0$ and therefore $\lim_{x\rightarrow a} -g(x) = 0$, then $\lim_{x\rightarrow a} f(x) = 0$ by the squeeze theorem.

If $\lim_{x\rightarrow a} g(x) = 3$, and $-g(x) \leq f(x) \leq g(x) \forall x$, then we can conclude that either $-3 \leq \lim_{x\rightarrow a} f(x) \leq 3$, or $\lim_{x\rightarrow a} f(x)$ doesn't exist.









