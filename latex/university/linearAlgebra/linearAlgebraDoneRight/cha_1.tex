\subsection{Vector Spaces}

For proofs, we're going to use a more general field $ \mathbb{F} $ to prove our theorems, so that our our proofs also apply to the fields $ \mathbb{R} $ and $ \mathbb{C} $. 

\begin{definition}
	$ \mathbb{F}^n $ is the set of all lists of length $ n $ of elements of $ \mathbb{F} $.
	\[
		\mathbb{F}^n = \{ \left( x_1, \dots, x_n \right) \; | \; x_j \in \mathbb{F} \; \forall j = 1, \dots, n \}
	\]
	For $ (x_1, \dots, x_n) \in \mathbb{F} $ and $ j \in \{ 1, \dots, n \} $, we say that $ x_j $ is the $ j^{th} $ \textbf{coordinate} of $ (x_1, \dots, x_n) $. $ (x_1, \dots, x_n) $ are scalars, a.k.a. numbers.
\end{definition}

This means that if we say $ \mathbb{F} = \mathbb{R} $ then we can apply the theorems in the real $ n $-space.

\subsubsection{Exercises 1.A}

\paragraph{1}

\begin{align*}
	\frac{1}{a+bi} = c+di \\
	(a+bi)(c+di) = 1 \\
	ac + adi + bci - bd = 1 \\
	(ac-bd) + (ad+bc)i = 1
\end{align*}

Which leads to

\begin{align*}
	ac-bd = 1 \\
	ad+bc = 0
\end{align*}

Because 1 has no imaginary part.

\begin{align*}
	c = \dfrac{1+bd}{a} \\
	d = \dfrac{-bc}{a}
\end{align*}

\begin{align*}
	c = \frac{1+b\frac{-bc}{a}}{a} \\
	c =\frac{1}{a} - \frac{b^2c}{a^2} \\
	c + \frac{b^2c}{a^2} = \frac{1}{a} \\
	c\left( 1 + \frac{b^2}{a^2} \right) = \frac{1}{a} \\
	c = \frac{1}{a\left( 1 + \frac{b^2}{a^2} \right)} \\
	c = \frac{1}{a + \frac{b^2}{a}} \\
	c = \frac{1}{\frac{a^2+b^2}{a}} \\
	c = \frac{a}{a^2+b^2}
\end{align*}





Now that we have $c$, we can pick out $d$ too.

\begin{align*}
	d = \frac{-bc}{a} \\
	d = \frac{-b\left( \frac{a}{a^2+b^2} \right)}{a} \\
	d = \frac{\frac{-ab}{a^2+b^2}}{a} \\
	d = \frac{-b}{a^2+b^2}
\end{align*}

So now we have $d$ and $c$ as real numbers, since $a$ and $b$ are real.

\begin{align*}
	c = \frac{a}{a^2+b^2} \\
	d = \frac{-b}{a^2+b^2}
\end{align*}

\paragraph{2}

\begin{align*}
	\frac{-1+\sqrt{3}i}{2} = \\
	\frac{-1}{2} + \frac{\sqrt{3}}{2}i = \\
	\cos \frac{2\pi}{3} + i\sin \frac{2\pi}{3} = \\
	e^{i\frac{2\pi}{3}}
\end{align*}

Which means that 

\begin{align*}
	\left( e^{i\frac{2\pi}{3}} \right)^3 = \\
	e^{i\frac{6\pi}{3}} = \\
	e^{i2\pi} = \\
	1
\end{align*}

Hence, the cube of $\frac{-1+\sqrt{3}i}{2}$ is equal to $1$, which was to be shown.

\subsubsection{Exercises 1.B}

\paragraph{Problem 1}

\begin{align*}
	-(-v) = v \\
\end{align*}

Add the additive inverse of $-(-v)$ to both sides

\begin{align*}
	0 = v+(-v)
\end{align*}

The additive inverse of $v$ added together with $v$ is the zero vector.

\begin{align*}
	0 = 0
\end{align*}

So the statement is true, which was to be proved.

\paragraph{Problem 2}

\begin{align*}
	av = 0
\end{align*}

Suppose that $a=0$ 





































































