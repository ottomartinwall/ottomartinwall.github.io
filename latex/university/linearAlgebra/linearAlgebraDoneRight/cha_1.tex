\subsection{Vector Spaces}

\subsubsection{Exercises 1.A}

\paragraph{1}

\begin{align*}
	\frac{1}{a+bi} = c+di \\
	(a+bi)(c+di) = 1 \\
	ac + adi + bci - bd = 1 \\
	(ac-bd) + (ad+bc)i = 1
\end{align*}

Which leads to

\begin{align*}
	ac-bd = 1 \\
	ad+bc = 0
\end{align*}

Because 1 has no imaginary part.

\begin{align*}
	c = \dfrac{1+bd}{a} \\
	d = \dfrac{-bc}{a}
\end{align*}

\begin{align*}
	c = \frac{1+b\frac{-bc}{a}}{a} \\
	c =\frac{1}{a} - \frac{b^2c}{a^2} \\
	c + \frac{b^2c}{a^2} = \frac{1}{a} \\
	c\left( 1 + \frac{b^2}{a^2} \right) = \frac{1}{a} \\
	c = \frac{1}{a\left( 1 + \frac{b^2}{a^2} \right)} \\
	c = \frac{1}{a + \frac{b^2}{a}} \\
	c = \frac{1}{\frac{a^2+b^2}{a}} \\
	c = \frac{a}{a^2+b^2}
\end{align*}





Now that we have $c$, we can pick out $d$ too.

\begin{align*}
	d = \frac{-bc}{a} \\
	d = \frac{-b\left( \frac{a}{a^2+b^2} \right)}{a} \\
	d = \frac{\frac{-ab}{a^2+b^2}}{a} \\
	d = \frac{-b}{a^2+b^2}
\end{align*}

So now we have $d$ and $c$ as real numbers, since $a$ and $b$ are real.

\begin{align*}
	c = \frac{a}{a^2+b^2} \\
	d = \frac{-b}{a^2+b^2}
\end{align*}
\dotfill

\paragraph{2}

\begin{align*}
	\frac{-1+\sqrt{3}i}{2} = \\
	\frac{-1}{2} + \frac{\sqrt{3}}{2}i = \\
	\cos \frac{2\pi}{3} + i\sin \frac{2\pi}{3} = \\
	e^{i\frac{2\pi}{3}}
\end{align*}

Which means that 

\begin{align*}
	\left( e^{i\frac{2\pi}{3}} \right)^3 = \\
	e^{i\frac{6\pi}{3}} = \\
	e^{i2\pi} = \\
	1
\end{align*}

Hence, the cube of $\frac{-1+\sqrt{3}i}{2}$ is equal to $1$, which was to be shown.

\subsubsection{Exercises 1.B}

\paragraph{Problem 1}

\begin{align*}
	-(-v) = v \\
\end{align*}

Add the additive inverse of $-(-v)$ to both sides

\begin{align*}
	0 = v+(-v) \\
	0 = 1v + (-1)v \\
	0 = (1 + (-1))v \\
	0v = 0
\end{align*}

Which is true because

\begin{align*}
	&0v = (0 + 0)v = 0v + 0v \\
\end{align*}

And if we then add the additive inverse of $0v$ to both sides and switch the sides around the equal sign we get

\begin{align*}
	0v = 0
\end{align*}

Which shows that $0v = 0$ and therefore $-(-v) = v$ is true for each $v \in V$.

So the statement is true, which was to be proved.

\paragraph{Problem 2}
\textbf{\textit{Suppose $a \in \mathbb{F}$, $v \in V$, and $av = 0$. Prove that $a = 0$ or $v = 0$.}}

Suppose that $a \neq 0$. We can then multiply both sides by the multiplicative inverse of $a$.

\begin{align*}
	v = 0\frac{1}{a} \\
	v = 0
\end{align*}

Suppose that $a = 0$. Since $0v = 0$ the value of $v$ doesn't matter for the statement to be true. 

In total, this means that $v = 0$ or $a = 0$, or both $v = 0$ and $a = 0$, which was to be proved.

\paragraph{Problem 3}
\textbf{\textit{Suppose $v,w \in V$. Explain why there exists a unique $x \in V$ such that $v + 3x = w$.}}

A unique $x$ exists in $V$ since $V$ is closed under addition and scaling and $x = (w-v)\frac{1}{3}$.

\paragraph{Problem 4} 
\textbf{\textit{The empty set is not a vector space. The empty set fails to satisfy only one of the requirements listed in 1.19. Which one?}}

The empty set, $\{\}$, doesn't satisfy the requirement of an additive identity. It has no elements so it must therefore not contain an additive identity.

\paragraph{Problem 5}
\textbf{\textit{Show that in the definition of a vector space (1.19), the additive inverse condition can be replaced with the condition that $$0v = 0, \forall v \in V.$$}}

\begin{exercise}
	fawf
\end{exercise}



























































