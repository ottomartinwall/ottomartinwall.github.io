\chapter{Exercises and My Solutions}

\newpage
\section{Analysis with an Introduction to Proof - Steven R. Lay}

\newpage
\subsection{Sets and Functions}

\subsubsection{Exercises 3}

\paragraph{(21)}
\textit{Suppose that $f:A\rightarrow B$ and let $C$ be a subset of $A$.
\begin{enumerate}
	\item Prove or give a counterexample: $f(A\backslash C)\subseteq f(A)\backslash f(C)$.
	\item Prove or give a counterexample: $f(A)\backslash f(C)\subseteq f(A\backslash C)$.
	\item What condition on $f$ will ensure that $f(A\backslash C) = f(A)\backslash f(C)$? Prove your answer.
	\item What condition of $f$ will ensure that $f(A\backslash C) = B\backslash f(C)$? Prove your answer.
\end{enumerate}}

\begin{proof}(\textbf{1})
	Suppose that $f(A\backslash C)\subseteq f(A)\backslash f(C)$.

	Let $x\in A\backslash C$, $x'\in C$ and $f(x)=f(x')$. Then, $f(x)\in f(A\backslash C)$, and therefore $f(x')\in f(A\backslash C)$. But since $f(x')\in f(C)$ and therefore $f(x)\in f(C)$, neither $f(x)$ or $f(x')$ is in $f(A)\backslash f(C)$. This contradicts our original statement because there exists a member in $f(A\backslash C)$ which is not in $f(A)\backslash f(C)$, so $f(A\backslash C)\nsubseteq f(A)\backslash f(C)$.
\end{proof}

\begin{proof}(\textbf{2})
	For any $y\in f(A)\backslash f(C)$, there exists an $x\in A$ such that $f(x)=y$. If $x\in C$, then $f(x)\in f(C)$ which means that $f(x)\neq y$, so by contradiction it must be that $x\notin C$. This implies that $x\in A\backslash C$, and therefore that $f(x)\in f(A\backslash C)$ and $y\in f(A\backslash C)$. Since $y\in f(A)\backslash f(C)$ implies that $y\in f(A\backslash C)$, the statement $f(A)\backslash f(C)\subseteq f(A\backslash C)$ must be true.
\end{proof}

\begin{proof}(\textbf{3})
	Proof 2 have already shown that $f(A)\backslash f(C)\subseteq f(A\backslash C)$, so to prove that $f(A)\backslash f(C) = f(A\backslash C)$ I must only prove the reverse of the first statement.
	
	Let $f$ be injective\footnote{this is the necessary condition such that $f(A\backslash C) = f(A)\backslash f(C)$.}. For any $y\in f(A\backslash C)$, there exists one and only one $x\in A\backslash C$ such that $f(x)=y$. Since $x\in A\backslash C$, $x\in A$ and $f(x)\in f(A)$. Also, since $x\in A\backslash C$, $x\notin C$ and $f(x)\notin f(C)$. This implies that $f(x)\in f(A)\backslash f(C)$ and thus $y\in f(A)\backslash f(C)$. Since $y\in f(A\backslash C)$ implies $y\in f(A)\backslash f(C)$, and $y\in f(A)\backslash f(C)$ implies $y\in f(A\backslash C)$ from proof 2, it must be that $f(A\backslash C) = f(A)\backslash f(C)$. 
\end{proof}

\begin{proof}(\textbf{4})
	Proof 3 in combination with that $f$ is surjective\footnote{Proof 3 needed the condition that $f$ was injective, and since proof 4 needs $f$ to be surjective and is based on proof 3, $f$ is now bijective.} means that $f(A\backslash C) = f(A)\backslash f(C) = B\backslash f(C)$ since $B =$ rng $f$ $= f(A)$. 
\end{proof}

\newpage
\paragraph{(32)}

\textit{Suppose that $f:A\rightarrow B$ is any function. Then a function $g:B\rightarrow A$ is called a
\begin{itemize}
	\item \textbf{left inverse} for $f$ if $g(f(x))=x$ for all $x\in A$,
	\item \textbf{right inverse} for $f$ if $f(g(y))=y$ for all $y\in B$.
\end{itemize}
\begin{enumerate}
	\item Prove that $f$ has a left inverse iff $f$ is injective. 
	\item Prove that $f$ has a right inverse iff $f$ is surjective.
\end{enumerate}}

\begin{proof}(\textbf{1})
	Suppose that $f$ is injective. Let $g = \{(b,a)\in B\times A:(a,b)\in f\}\cup\{(b,a)\in B\times A:b\notin f(A)\}$\footnote{I added the part $\cup\{(b,a)\in B\times A:b\notin f(A)\}$ to $g$ to show that $f$ must not be surjective.}. By definition, each $a\in A$ corresponds to one and only one $b\in B$ such that $f(a)=b$, and because of the definition of $g$, for each $b\in B$ such that $f(a)=b$, $g(b)=a$, which implies that $g(f(a))=a$ for all $a\in A$. 
	
	Conversely, suppose that $f(x)\in B$ and $f(x')\in B$, and that $f(x)=f(x')$. If $g(f(a))=a$ for all $a\in A$, $g(f(x))=g(f(x'))$ implies that $x=x'$. Therefore, $f$ is injective.
\end{proof}

\begin{proof}(\textbf{2})
	Suppose that $f$ has a right inverse and therefore $f(g(y))=y$ for all $y\in B$. This implies that $f$ is surjective, since for all $y\in B$ there exists some $x\in A$, which may be $g(y)$, such that $f(x)=y$.
\end{proof}

\newpage
\paragraph{(33)}

\textit{Let $S$ be a nonempty set and let $F$ be the set of all functions that map $S$ into $S$. Suppose that for every $f$ and $g$ in $F$ we have
\begin{align*}
	(f\circ g)(x)=(g\circ f)(x), \forall x\in S
\end{align*}
Prove that $S$ has only one element.}

\begin{proof}
	If $S$ contains more than one element, then there exists some functions $f$ and $g$ in $F$ that are neither surjective nor injective. Suppose that $x,x'\in S$ and that $x\neq x'$, and that $f(x)=x'$ and $f(x')=x'$, and that $g(x)=x$ and $g(x')=x$. Then $f(g(x))=f(x)=x'$, and $g(f(x))=g(x')=x$, which contradicts the statement that $(f\circ g)(x)=(g\circ f)(x), \forall x\in S$, so $S$ must contain less than two elements. Since $S$ is nonempty, it must therefore contain one element.
\end{proof}
















































































