\newpage
\chapter{Set Theory}

\section{Ordered Pairs}

\begin{definition}[Ordered Pair]
	The \textbf{ordered pair} $(a,b)$ is the set whose members are $\{a\}$ and $\{a,b\}$. In symbols we have
	\begin{align*}
		(a,b) = \{\{a\}, \{a,b\}\}
	\end{align*}
\end{definition}

This definition ensures that order matters. To show this, this theorem and its proof should suffice.

\begin{theorem}[Ordered Pair Theorem]
	\footnote{this is a made up name by me}
	\begin{align*}
		(a,b)=(c,d) \leftrightarrow a=c, b=d
	\end{align*}
\end{theorem}

\begin{proof}
	If $a=c$ and $b=d$, then 
	\begin{align*}
		(a,b)=\{\{a\}, \{a,b\} = \{\{c\}, \{c,d\}\}=(c,d)
	\end{align*}
	Conversely, suppose that $(a,b)=(c,d)$. Then by our definition we have $\{\{a\}, \{a,b\}\} = \{\{c\}, \{c,d\}\}$. We wish to conclude that $a=c$ and $b=d$. To this end we consider two cases, depending on whether $a=b$ or $a\neq b$.
	
	If $a=b$, then $\{a\} = \{a,b\}$, so $(a,b) = \{\{a\}\}$. Since $(a,b)=(c,d)$, we then have 
	\begin{align*}
		\{\{a\}\} = \{\{c\}, \{c,d\}\}.
	\end{align*}
	The set on the left has only one member, $\{a\}$. Thus the set on the right can have only one member, so $\{c\} = \{c,d\}$, and we can conclude that $c=d$. But then $\{\{a\}\} = \{\{c\}\}$, so $\{a\} = \{c\}$ and $a=c$. Thus $a=b=c=d$.
	
	On the other hand, if $a\neq b$, then from the preceding argument it follows that $c\neq d$. Since $(a,b) = (c,d)$, we must have 
	\begin{align*}
		\{a\} \in \{\{c\}, \{c,d\}\},
	\end{align*}
	which means that $\{a\} = \{c\}$ or $\{a\} = \{c,d\}$. In either case we have $c \in \{a\}$, so $a=c$. Again, since $(a,b)=(c,d)$, we must also have
	\begin{align*}
		\{a,b\} \in \{\{c\}, \{c,d\}\}.
	\end{align*}
	Thus $\{a,b\} = \{c\}$ or $\{a,b\} = \{c,d\}$. But $\{a,b\}$ has two distinct members and $\{c\}$ has only one, so we must have $\{a,b\} = \{c,d\}$. Now $a=c$, $a\neq b$, and $b\in \{c,d\}$, which implies that $b=d$. 
\end{proof}

\begin{definition}[Cartesian Product]
	If $A$ and $B$ are sets, then the \textbf{Cartesian product} (or \textbf{cross product}) of $A$ and $B$, written $A\times B$, is the set of all ordered pairs $(a,b)$ such that $a\in A$ and $b\in B$. In symbols,
	\begin{align*}
		A\times B = \{(a,b):(a\in A) \land (b\in B)\}.
	\end{align*}
\end{definition}

\newpage
\section{Relation}

\begin{definition}[Relation]
	Let $A$ and $B$ be sets. A \textbf{relation between $A$ and $B$} is any subset R of $A\times B$. We say that an element $a$ in $A$ is \textbf{related} by R to an element $b$ in $B$ if $(a,b)\in \text{R}$, and we often denote this by writing "$a\text{R}b$". The first set $A$ is referred to as the \textbf{domain} of the relation and denoted by dom R. If $B=A$, then we speak of a relation R $\subseteq A\times A$ being a \textbf{relation on A}.
\end{definition}

\begin{definition}[Equivalence Relation]
	A relation R on a set $S$ is an \textbf{equivalence relation} if it has the following properties for all $x,y,z \in S$:
	\begin{itemize}
		\item \textbf{Reflexive property:} $x$R$x$
		\item \textbf{Symmetric property:} $x$R$y$ $\leftrightarrow$ $y$R$x$
		\item \textbf{Transitive property:} ($x$R$y$ $\land$ $y$R$z$) $\rightarrow$ $x$R$z$
	\end{itemize}
\end{definition}

An example for a \textbf{equivalence relation} is the relation "is parallel to" when considering all lines in the plane, if we agree that a line is parallel to itself.

\begin{definition}[Equivalence Class]
	\label{Equivalence Class}
	Given an equivalence relation R on a set $S$, the \textbf{equivalence class} with respect to R of $x\in S$ is the set
	\begin{align*}
		E_x = \{y\in S:y\text{R}x\}
	\end{align*}
\end{definition}

\begin{eg}
 	Let $S = \{a:a\text{ lives in Sweden}\}$, which is the set of all people living in Sweden. Also, let a equivalence relation on this set be
 	\begin{align*}
		R = \{(a,b)\in S\times S:a \text{ was born in the same year as } b\}.
	\end{align*}
	Then
 	\begin{align*}
		E_x = \{y\in S:y\text{R}x\}
	\end{align*}
	is the set of all people living in Sweden who was born during the same year as some person $x$ who is also living in Sweden.
 \end{eg}
 
\begin{theorem}
	Two equivalence classes on the same set $S$ with the same equivalence relation R must be disjoint or equal.
\end{theorem}

\begin{proof}
	Let R be an equivalence relation on a set $S$, and let $E_x$ and $E_y$ be two equivalence classes with respect to R of $x\in S$. Suppose that they overlap, then there exists some $w\in E_x \cap E_y$. For all $x'\in E_x$ we have $x'$R$x$, and because $w\in E_x$, $w$R$x$, and by symmetry, $x$R$w$. Also, $w\in E_y$ so $w$R$y$. By using transitivity, $x'$R$x$ and $x$R$w$ and $w$R$y$ implies that $x'$R$y$, which means that $x'\in E_y$ and that $E_x \subseteq E_y$. 
	
	Conversely, for all $y'\in E_y$ we have $y'$R$y$, and because $w\in E_y$, $w$R$y$, and by the symmetry property, $y$R$w$. Also, $w\in E_x$ so $w$R$x$. By using the transitivity property, $y'$R$y$ and $y$R$w$ and $w$R$x$ implies that $y'$R$x$ and that $E_y \subseteq E_x$. Since $E_x \subseteq E_y$ and $E_x \supseteq E_y$, it must be that $E_y = E_x$.
\end{proof}

\begin{definition}
	A \textbf{partition} of a set $S$ is a collection P of nonempty subsets of $S$ such that
	\begin{itemize}
		\item Each $x\in S$ belongs to some subset $A\in$ P.
		\item For all $A,B\in$ P, if $A\neq B$, then $A\cap B = \emptyset$.
	\end{itemize}
	A member of P is called a \textbf{piece} of the partition.
\end{definition}

\begin{eg}
	Two equivalence classes on the same set $S$ with the same equivalence relation R who are not equal (and therefore disjoint) are two pieces of a partition P on the set $S$.
\end{eg}

\newpage
\section{Cardinality}

\textbf{This subsection requires understanding of the definition of a function between two sets, and understanding of surjection and injection (and therefore bijection). This can be learned in Chapter \ref{ChapterFunctions}}.

\begin{definition}[Set Equivalence]
	Two sets $S$ and $T$ are called \textbf{set equivalent}, and we write $S\sim T$, if there exists a bijective function from $S$ onto $T$.
\end{definition}

This definition ensures that if two sets are set equivalent, they contain the same number of elements, since a bijective function between them will set up a one-to-one correspondence between the elements of each set.

\begin{definition}[Finite or Infinite Set]
	A set $S$ is said to be \textbf{finite} if $S=\emptyset$ or if there exists $n\in \N$ and a bijection $f:\{1,2,\hdots,n\}\rightarrow S$.\footnote{Moving forward, we will make use of the set $I_n=\{1,2,\hdots,n\}$.} If a set is not finite, it is said to be \textbf{infinite}.
\end{definition}

\begin{definition}
	The \textbf{cardinal number} of the set $I_n=\{1,2,\hdots,n\}$ is $n$, and if $S\sim I_n$, we say that $S$ \textbf{has n elements}. The cardinal number of $\emptyset$ is taken to be $0$. If a cardinal number is not finite, it is called \textbf{transfinite}. 
\end{definition}

\begin{definition}
	A set $S$ is said to be \textbf{denumerable} if there exists a bijection $f:\N \rightarrow S$. If a set is finite or denumerable, it is called \textbf{countable}. If a set is not countable, it is \textbf{uncountable}. The cardinal number of a denumerable set is denoted by $\aleph_0$.
\end{definition}

\begin{remark}
	Against our intuition from finite sets, if $E$ is the set of all even natural numbers, then $\N\sim E$, because if $f(n)=2n$, then $f:\N\rightarrow E$ is bijective. Therefore, both $\N$ and $E$ has the cardinal number $\aleph_0$ even though $E\subset \N$.
\end{remark}

\begin{eg}
	$\Z$, the set of all integers, is denumerable since $f:\N \rightarrow \Z$ is bijective if $$f(n)=\begin{cases}
		0 \text{ if $n=1$}\\
		\frac{n}{2} \text{ if $n$ is even}\\
		\lceil-\frac{n}{2}\rceil \text{ if $n$ is odd}
	\end{cases}$$
	because this leads to that
	\begin{align*}
		f(1) &\rightarrow 0 \\
		f(2) &\rightarrow 1 \\
		f(3) &\rightarrow (-1) \\
		f(4) &\rightarrow 2 \\
		f(5) &\rightarrow (-2) \\
		&\vdots
	\end{align*}
	So for any $b\in \Z$, there exists a $a\in \N$ such that $f(a)=b$, which implies that $f$ is surjective, and there is also a one to one correspondence between the two sets so $f$ is injective, and therefore bijective.
\end{eg}

\begin{notation}
	For any nonempty finite set $S$, there exists a bijection $f:I_n\rightarrow S$ for some $n\in \N$. Therefore, we use this function to count the members as $f(1),f(2),f(3),\hdots,f(n)$. Letting $f(k)=s_k$ we can write $S=\{s_1,s_2,\hdots,s_n\}$. We can also do this for any denumerable set $T$, since because it is denumerable, there exists a bijection $g:\N\rightarrow T$, so we can use $g(k)=t_k$ to write $T=\{t_1,t_2,t_3,\hdots\}$.
\end{notation}

\begin{lemma}
	\label{lemmalabel1}
	Every subset of a finite set is finite.
\end{lemma}

\begin{proof}
	--- NOT DONE
	% TODO proof, Set Theory
\end{proof}


\begin{theorem}
	\label{theolabel3}
	Let $S$ be a countable set and let $T\subseteq S$. Then $T$ is countable.
\end{theorem}

\begin{proof}
	If $T$ is finite, then we are done. Thus we may assume that $T$ is infinite. This implies that $S$ is infinite\footnote{This implication is true by lemma \ref{lemmalabel1}}, so $S$ is denumerable (since it is countable and infinite). Therefore, there exists a bijection $f:\N\rightarrow S$ and we can write $S$ as a list of distinct members
	\begin{align*}
		S=\{s_1,s_2,s_3,\hdots\}
	\end{align*}
	where $f(n)=s_n$. Now let
	\begin{align*}
		A=\{n\in\N:s_n\in T\}.
	\end{align*}
	Since $A$ is a nonempty subset of $\N$, the \textit{Well-Ordering Property of $\N$} implies that $A$ has a least member, say $a_1$. Similarly, the set $A\backslash \{a_1\}$ has a least member, say $a_2$. In general, having chosen $a_1,\hdots,a_k$, let $a_{k+1}$ be the least member in $A\backslash \{a_1,\hdots,a_k\}$. Essentially, if we select from our listing of $S$ those terms that are in $T$ and keep them in the same order, then $a_n$ is the subscript of the \textit{n}th term in this new list.
	
	Now define a function $g:\N\rightarrow\N$ by $g(n)=a_n$. Since $T$ is infinite, $g$ is defined for every $n\in\N$. Since $a_{n+1}\notin \{a_1,\hdots,a_n\}$, g must be injective\footnote{I suppose that this is a small proof by induction that $g$ is injective? This proof is not mine and is taken from \textit{Analysis with an Introduction to Proof}.}. Thus tje composition $f\circ g$ is also injective. Since each element of $T$ is somewhere in the listing of $S$, $g(\N)$ includes all the subscripts of terms in $T$. Thus $f\circ g$ is a bijection from $\N$ onto $T$ and $T$ is denumerable.
\end{proof}

\begin{theorem}
	Let S be a nonempty set. The following three conditions are equivalent.
	\begin{enumerate}
		\item $S$ is countable.
		\item There exists an injection $f:S\rightarrow \N$.
		\item There exists a surjection $g:\N\rightarrow S$.
	\end{enumerate}
\end{theorem}

\begin{proof}
	Suppose that $S$ is countable. Then there exists some bijection $h:J\rightarrow S$ where $J=I_n$ for some $n\in \N$ if $S$ is finite, or $J=\N$ if $S$ is infinite. In either case, $h^{-1}:S\rightarrow \N$ is at least injective. Thus (1) implies (2). 
	
	Now suppose that there exists an injection $f:S\rightarrow \N$. Then $f$ is a bijection from $S$ to $f(S)$, so $f^{-1}$ is a bijection from $f(S)$ to $S$. Let $g:\N\rightarrow S$ be defined by
	\begin{align*}
		g(n) = \begin{cases}
			f^{-1}(n), \text{ if $n\in f(S)$} \\
			p, \text{ if $n\notin f(S)$}
		\end{cases}
	\end{align*}
	where $p\in S$. Then $g[f(S)]=f^{-1}[f(S)]=S$ and $g[\N\backslash f(S)]=\{p\}$, so that $g$ is a surjection from $\N$ onto $S$. Thus, (2) implies (3).
	
	Finally, suppose that there exists a surjection $g:\N\rightarrow S$. Define $h:S\rightarrow \N$ by
	\begin{align*}
		h(s) \text{ is the smallest $n\in \N$ such that $g(n)=s$.}
	\end{align*}
	Then $h$ is an injection from $S$ to $\N$, and hence a bijection from $S$ onto the subset $h(S)$ of $\N$. Since $\N$ is countable, theorem \ref{theolabel3} implies that $h(S)$ is countable. Since $S$ and $h(S)$ are set equivalent, because there exists a bijection between the two sets, $S$ is also countable.
\end{proof}

\begin{theorem}
	The set $\R$ of real numbers is uncountable.
\end{theorem}

\begin{proof}
	Since any subset of a countable set is countable (theorem \ref{theolabel3}), it suffices to show that the interval $J=(0,1)$ is uncountable. If $J$ were countable, we could list its members and have 
	$$J=\{x_1,x_2,x_3,\hdots\}=\{x_n:n\in\N\}.$$
	Each element of $J$ has an infinite decimal expansion, so we can write
	\begin{align*}
		x_1 &= 0.a_{11}a_{12}a_{13}\hdots, \\
		x_2 &= 0.a_{21}a_{22}a_{23}\hdots, \\
		x_3 &= 0.a_{31}a_{32}a_{33}\hdots, \\
		&\vdots
	\end{align*}
	where each $a_{ij}\in \{0,1,\hdots,9\}$. We now construct a real number $y=b_1b_2b_3\hdots$ by defining
	$$b_n=\begin{cases}
		2, \quad \text{if $a_{nn}\neq 2$} \\
		3, \quad \text{if $a_{nn}=2$}
	\end{cases}$$
	Since each digit in the decimal expansion of $y$ is either $2$ or $3$, $y\in J$. But $y$ is not one of the numbers $x_n$, since it differs from $x_n$ in the $n$th decimal place. This contradicts our assumption that $J$ is countable, so $J$ must be uncountable.
\end{proof}

\begin{definition}[Cardinal Number of a Set]
	We denote the cardinal number of a set $S$ by $|S|$, so that we have $|S|=|T|$ iff $S$ and $T$ are set equivalent, which implies that there exists a bijection $f:S\rightarrow T$. We define $|S|\leq|T|$ to mean that there exists an injection $f:S\rightarrow T$, and $|S|<|T|$ means that $|S|\leq|T|$ and $|S|\neq|T|$.
\end{definition}

\begin{theorem}
	If $S\subseteq T$, then $\abs{S}\leq \abs{T}$.
\end{theorem}

\begin{proof}(\textbf{1})
	If $S\subseteq T$, then for each $s\in S$ there exists one $t\in T$ with the relation $s=t$. If we let a function $f:S\rightarrow T$ be defined by $f(s)=s$, it is injective, and since there exists an injection that maps $S$ into $T$, we say that $\abs{S}\leq \abs{T}$ by definition.
\end{proof}

\begin{remark}
	$\abs{\R}$ is usually written as $c$, for continuum. Since $\N\subseteq\R$, we have $\aleph_0\leq c$ by the theorem above. In fact, since $\N$ is countable and $\R$ is uncountable, we have $\aleph_0<c$. Therefore, there exists more than one transfinite cardinal number.
\end{remark}

\begin{definition}[Power Set]
	For any set $S$, $\curs{P}(S)$ is the collection of all subsets of $S$. This collection is called the \textbf{power set} of $S$.
\end{definition}

\begin{theorem}
	For any set $S$, we have $\abs{S}<\abs{\curs P(S)}$
\end{theorem}

\begin{proof}
	The function $g:S\rightarrow \curs P(S)$ given by $g(s)=\{s\}$ is injective, so we have $\abs{S}\leq \abs{\curs P(S)}$. To prove that $\abs{S}\neq \abs{\curs P(S)}$, we show that no function from $S$ to $\curs P(S)$ can be surjective\footnote{This ensures that no function can be bijective from $S$ to $\curs P(S)$, so that $\abs{S}\neq \abs{P(S)}$}. Suppose that $f:S\rightarrow \curs P(S)$. Then for each $x\in S$, $f(x)\subseteq S$. For some $x$ in $S$ it may be that $x\in f(x)$, or $x\notin f(x)$. Let 
	$$T=\{x\in S:x\notin f(x)\}.$$
	Then $T\subseteq S$, so $T\in \curs P(S)$. If $f$ were surjective, then $T=f(y)$ for some $y\in S$. Now either $y\in T$ or $y\notin T$. If $y\in T$, then by the definiton of $T$, $y\notin T$. If $y\notin T$, then by the definition of $T$, $y\in T$. Therefore, $y\in T$ iff $y\notin T$, which is a contradiction, so it must be that $\abs{S}<\abs{\curs P(S)}$.
\end{proof}
















































